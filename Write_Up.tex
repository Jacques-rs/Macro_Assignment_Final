\documentclass[11pt,preprint, authoryear]{elsarticle}

\usepackage{lmodern}
%%%% My spacing
\usepackage{setspace}
\setstretch{1.2}
\DeclareMathSizes{12}{14}{10}{10}

% Wrap around which gives all figures included the [H] command, or places it "here". This can be tedious to code in Rmarkdown.
\usepackage{float}
\let\origfigure\figure
\let\endorigfigure\endfigure
\renewenvironment{figure}[1][2] {
    \expandafter\origfigure\expandafter[H]
} {
    \endorigfigure
}

\let\origtable\table
\let\endorigtable\endtable
\renewenvironment{table}[1][2] {
    \expandafter\origtable\expandafter[H]
} {
    \endorigtable
}


\usepackage{ifxetex,ifluatex}
\usepackage{fixltx2e} % provides \textsubscript
\ifnum 0\ifxetex 1\fi\ifluatex 1\fi=0 % if pdftex
  \usepackage[T1]{fontenc}
  \usepackage[utf8]{inputenc}
\else % if luatex or xelatex
  \ifxetex
    \usepackage{mathspec}
    \usepackage{xltxtra,xunicode}
  \else
    \usepackage{fontspec}
  \fi
  \defaultfontfeatures{Mapping=tex-text,Scale=MatchLowercase}
  \newcommand{\euro}{€}
\fi

\usepackage{amssymb, amsmath, amsthm, amsfonts}

\def\bibsection{\section*{References}} %%% Make "References" appear before bibliography


\usepackage[round]{natbib}

\usepackage{longtable}
\usepackage[margin=2.3cm,bottom=2cm,top=2.5cm, includefoot]{geometry}
\usepackage{fancyhdr}
\usepackage[bottom, hang, flushmargin]{footmisc}
\usepackage{graphicx}
\numberwithin{equation}{section}
\numberwithin{figure}{section}
\numberwithin{table}{section}
\setlength{\parindent}{0cm}
\setlength{\parskip}{1.3ex plus 0.5ex minus 0.3ex}
\usepackage{textcomp}
\renewcommand{\headrulewidth}{0.2pt}
\renewcommand{\footrulewidth}{0.3pt}

\usepackage{array}
\newcolumntype{x}[1]{>{\centering\arraybackslash\hspace{0pt}}p{#1}}

%%%%  Remove the "preprint submitted to" part. Don't worry about this either, it just looks better without it:
\makeatletter
\def\ps@pprintTitle{%
  \let\@oddhead\@empty
  \let\@evenhead\@empty
  \let\@oddfoot\@empty
  \let\@evenfoot\@oddfoot
}
\makeatother

 \def\tightlist{} % This allows for subbullets!

\usepackage{hyperref}
\hypersetup{breaklinks=true,
            bookmarks=true,
            colorlinks=true,
            citecolor=blue,
            urlcolor=blue,
            linkcolor=blue,
            pdfborder={0 0 0}}


% The following packages allow huxtable to work:
\usepackage{siunitx}
\usepackage{multirow}
\usepackage{hhline}
\usepackage{calc}
\usepackage{tabularx}
\usepackage{booktabs}
\usepackage{caption}


\newenvironment{columns}[1][]{}{}

\newenvironment{column}[1]{\begin{minipage}{#1}\ignorespaces}{%
\end{minipage}
\ifhmode\unskip\fi
\aftergroup\useignorespacesandallpars}

\def\useignorespacesandallpars#1\ignorespaces\fi{%
#1\fi\ignorespacesandallpars}

\makeatletter
\def\ignorespacesandallpars{%
  \@ifnextchar\par
    {\expandafter\ignorespacesandallpars\@gobble}%
    {}%
}
\makeatother

\newlength{\cslhangindent}
\setlength{\cslhangindent}{1.5em}
\newenvironment{CSLReferences}%
  {\setlength{\parindent}{0pt}%
  \everypar{\setlength{\hangindent}{\cslhangindent}}\ignorespaces}%
  {\par}


\urlstyle{same}  % don't use monospace font for urls
\setlength{\parindent}{0pt}
\setlength{\parskip}{6pt plus 2pt minus 1pt}
\setlength{\emergencystretch}{3em}  % prevent overfull lines
\setcounter{secnumdepth}{5}

%%% Use protect on footnotes to avoid problems with footnotes in titles
\let\rmarkdownfootnote\footnote%
\def\footnote{\protect\rmarkdownfootnote}
\IfFileExists{upquote.sty}{\usepackage{upquote}}{}

%%% Include extra packages specified by user

%%% Hard setting column skips for reports - this ensures greater consistency and control over the length settings in the document.
%% page layout
%% paragraphs
\setlength{\baselineskip}{12pt plus 0pt minus 0pt}
\setlength{\parskip}{12pt plus 0pt minus 0pt}
\setlength{\parindent}{0pt plus 0pt minus 0pt}
%% floats
\setlength{\floatsep}{12pt plus 0 pt minus 0pt}
\setlength{\textfloatsep}{20pt plus 0pt minus 0pt}
\setlength{\intextsep}{14pt plus 0pt minus 0pt}
\setlength{\dbltextfloatsep}{20pt plus 0pt minus 0pt}
\setlength{\dblfloatsep}{14pt plus 0pt minus 0pt}
%% maths
\setlength{\abovedisplayskip}{12pt plus 0pt minus 0pt}
\setlength{\belowdisplayskip}{12pt plus 0pt minus 0pt}
%% lists
\setlength{\topsep}{10pt plus 0pt minus 0pt}
\setlength{\partopsep}{3pt plus 0pt minus 0pt}
\setlength{\itemsep}{5pt plus 0pt minus 0pt}
\setlength{\labelsep}{8mm plus 0mm minus 0mm}
\setlength{\parsep}{\the\parskip}
\setlength{\listparindent}{\the\parindent}
%% verbatim
\setlength{\fboxsep}{5pt plus 0pt minus 0pt}



\begin{document}



\begin{frontmatter}  %

\title{Corruption in the 19th Century Cape Colony}

% Set to FALSE if wanting to remove title (for submission)




\author[Add1]{Samantha Scott}
\ead{20945043@sun.ac.za}





\address[Add1]{Stellenbosch University, Cape Town, South Africa}

\cortext[cor]{Corresponding author: Samantha Scott}

\begin{abstract}
\small{
Abstract to be written here. The abstract should not be too long and
should provide the reader with a good understanding what you are writing
about. Academic papers are not like novels where you keep the reader in
suspense. To be effective in getting others to read your paper, be as
open and concise about your findings here as possible. Ideally, upon
reading your abstract, the reader should feel he / she must read your
paper in entirety.
}
\end{abstract}

\vspace{1cm}


\begin{keyword}
\footnotesize{
Multivariate GARCH \sep Kalman Filter \sep Copula \\
\vspace{0.3cm}
}
\footnotesize{
\textit{JEL classification} L250 \sep L100
}
\end{keyword}



\vspace{0.5cm}

\end{frontmatter}



%________________________
% Header and Footers
%%%%%%%%%%%%%%%%%%%%%%%%%%%%%%%%%
\pagestyle{fancy}
\chead{}
\rhead{}
\lfoot{}
\rfoot{\footnotesize Page \thepage}
\lhead{}
%\rfoot{\footnotesize Page \thepage } % "e.g. Page 2"
\cfoot{}

%\setlength\headheight{30pt}
%%%%%%%%%%%%%%%%%%%%%%%%%%%%%%%%%
%________________________

\headsep 35pt % So that header does not go over title




\hypertarget{introduction}{%
\section{Introduction}\label{introduction}}

The following paper aims to establish whether the 19th century Cape
Colony was corrupt or not, as well as the extent of this corruption

\hypertarget{historical-background}{%
\section{Historical Background}\label{historical-background}}

The Civil Service Act of 1885 as amended by Act 31 of 1888 gives an
indication that civil servants are not allowed to hold any office for
profit during active service Moving forward, it is important to
understand why this scenario is considered as corruption. I have
included a short background. However, there isn't much literature on
this topic, so the background is sparce. I start with a brief definition
of corruption, which is considered to be the abuse of a trusted position
with the purpose of illegally obtaining benefit. It is difficult to say
when corruption in SA started, due to minimal written records. However,
what literature has claimed, is that Jan van Riebeck (the founding
father of colonial SA), was accused of corruption. So, with a foundation
of corruption, it becomes important to question the pattern and extent
of corruption in SA The next question is to consider why it would be
considered as corruption if an individual is a private shareholder and a
civil servant simultaneously? According to the Civil Service Act of 1885
as amended by Act 31 of 1888, civil servants may not hold office for
profit during their time in active service. This is because civil
servants may Influence the regulation of private companies for their own
benefit, as they have something to gain.

\hypertarget{data}{%
\section{Data}\label{data}}

The data used in the investigation is a list of civil servants as well
as a list of limited liability companies. In order to fulfil this goal,
I make use of two datasets, namely a list of civil servants and a list
of shareholders of limited liability companies -- both from the 19th
century. The JscR dataset has around 7 000 observations, and the civil
servant list has around 40 000 observations.

\hypertarget{methodology}{%
\section{Methodology}\label{methodology}}

To establish whether there was corruption, matches between names in the
datasets needs to be established. 1. Concatenate name vectors in each of
the datasets -- create a full name variable 2. Remove spaces and dots
between 1st, 2nd, and last names -- ensure the highest number of matches
possible. Any difference in spaces or punctuation will negatively impact
the number of matches 3. Match names by doing an ``inner join'' R and
remove any duplicates 4. Aggregate number of shares held in each office
5. Use previous literature to establish LR impact

\hypertarget{results-and-discussion}{%
\section{Results and Discussion}\label{results-and-discussion}}

After matching the names, it is evident that there was corruption in the
18th century Cape Colony. The most problematic office was the Divisional
Courts and Offices. The industry invested most in was

\hypertarget{limitations}{%
\section{Limitations}\label{limitations}}

The way in which names are written in the data sets varies. The one
dataset contains more initials, whereas majority of the names in the
other dataset are written out in full.

More common names are more likely to produce matches.

\hypertarget{conclusion}{%
\section{Conclusion}\label{conclusion}}

\hypertarget{reference-list}{%
\section{Reference List}\label{reference-list}}

\bibliography{Tex/ref}





\end{document}
