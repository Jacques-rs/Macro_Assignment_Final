\documentclass[11pt,preprint, authoryear]{elsarticle}

\usepackage{lmodern}
%%%% My spacing
\usepackage{setspace}
\setstretch{1.2}
\DeclareMathSizes{12}{14}{10}{10}

% Wrap around which gives all figures included the [H] command, or places it "here". This can be tedious to code in Rmarkdown.
\usepackage{float}
\let\origfigure\figure
\let\endorigfigure\endfigure
\renewenvironment{figure}[1][2] {
    \expandafter\origfigure\expandafter[H]
} {
    \endorigfigure
}

\let\origtable\table
\let\endorigtable\endtable
\renewenvironment{table}[1][2] {
    \expandafter\origtable\expandafter[H]
} {
    \endorigtable
}


\usepackage{ifxetex,ifluatex}
\usepackage{fixltx2e} % provides \textsubscript
\ifnum 0\ifxetex 1\fi\ifluatex 1\fi=0 % if pdftex
  \usepackage[T1]{fontenc}
  \usepackage[utf8]{inputenc}
\else % if luatex or xelatex
  \ifxetex
    \usepackage{mathspec}
    \usepackage{xltxtra,xunicode}
  \else
    \usepackage{fontspec}
  \fi
  \defaultfontfeatures{Mapping=tex-text,Scale=MatchLowercase}
  \newcommand{\euro}{€}
\fi

\usepackage{amssymb, amsmath, amsthm, amsfonts}

\def\bibsection{\section*{References}} %%% Make "References" appear before bibliography


\usepackage[round]{natbib}

\usepackage{longtable}
\usepackage[margin=2.3cm,bottom=2cm,top=2.5cm, includefoot]{geometry}
\usepackage{fancyhdr}
\usepackage[bottom, hang, flushmargin]{footmisc}
\usepackage{graphicx}
\numberwithin{equation}{section}
\numberwithin{figure}{section}
\numberwithin{table}{section}
\setlength{\parindent}{0cm}
\setlength{\parskip}{1.3ex plus 0.5ex minus 0.3ex}
\usepackage{textcomp}
\renewcommand{\headrulewidth}{0.2pt}
\renewcommand{\footrulewidth}{0.3pt}

\usepackage{array}
\newcolumntype{x}[1]{>{\centering\arraybackslash\hspace{0pt}}p{#1}}

%%%%  Remove the "preprint submitted to" part. Don't worry about this either, it just looks better without it:
\makeatletter
\def\ps@pprintTitle{%
  \let\@oddhead\@empty
  \let\@evenhead\@empty
  \let\@oddfoot\@empty
  \let\@evenfoot\@oddfoot
}
\makeatother

 \def\tightlist{} % This allows for subbullets!

\usepackage{hyperref}
\hypersetup{breaklinks=true,
            bookmarks=true,
            colorlinks=true,
            citecolor=blue,
            urlcolor=blue,
            linkcolor=blue,
            pdfborder={0 0 0}}


% The following packages allow huxtable to work:
\usepackage{siunitx}
\usepackage{multirow}
\usepackage{hhline}
\usepackage{calc}
\usepackage{tabularx}
\usepackage{booktabs}
\usepackage{caption}


\newenvironment{columns}[1][]{}{}

\newenvironment{column}[1]{\begin{minipage}{#1}\ignorespaces}{%
\end{minipage}
\ifhmode\unskip\fi
\aftergroup\useignorespacesandallpars}

\def\useignorespacesandallpars#1\ignorespaces\fi{%
#1\fi\ignorespacesandallpars}

\makeatletter
\def\ignorespacesandallpars{%
  \@ifnextchar\par
    {\expandafter\ignorespacesandallpars\@gobble}%
    {}%
}
\makeatother

\newlength{\cslhangindent}
\setlength{\cslhangindent}{1.5em}
\newenvironment{CSLReferences}%
  {\setlength{\parindent}{0pt}%
  \everypar{\setlength{\hangindent}{\cslhangindent}}\ignorespaces}%
  {\par}


\urlstyle{same}  % don't use monospace font for urls
\setlength{\parindent}{0pt}
\setlength{\parskip}{6pt plus 2pt minus 1pt}
\setlength{\emergencystretch}{3em}  % prevent overfull lines
\setcounter{secnumdepth}{5}

%%% Use protect on footnotes to avoid problems with footnotes in titles
\let\rmarkdownfootnote\footnote%
\def\footnote{\protect\rmarkdownfootnote}
\IfFileExists{upquote.sty}{\usepackage{upquote}}{}

%%% Include extra packages specified by user

%%% Hard setting column skips for reports - this ensures greater consistency and control over the length settings in the document.
%% page layout
%% paragraphs
\setlength{\baselineskip}{12pt plus 0pt minus 0pt}
\setlength{\parskip}{12pt plus 0pt minus 0pt}
\setlength{\parindent}{0pt plus 0pt minus 0pt}
%% floats
\setlength{\floatsep}{12pt plus 0 pt minus 0pt}
\setlength{\textfloatsep}{20pt plus 0pt minus 0pt}
\setlength{\intextsep}{14pt plus 0pt minus 0pt}
\setlength{\dbltextfloatsep}{20pt plus 0pt minus 0pt}
\setlength{\dblfloatsep}{14pt plus 0pt minus 0pt}
%% maths
\setlength{\abovedisplayskip}{12pt plus 0pt minus 0pt}
\setlength{\belowdisplayskip}{12pt plus 0pt minus 0pt}
%% lists
\setlength{\topsep}{10pt plus 0pt minus 0pt}
\setlength{\partopsep}{3pt plus 0pt minus 0pt}
\setlength{\itemsep}{5pt plus 0pt minus 0pt}
\setlength{\labelsep}{8mm plus 0mm minus 0mm}
\setlength{\parsep}{\the\parskip}
\setlength{\listparindent}{\the\parindent}
%% verbatim
\setlength{\fboxsep}{5pt plus 0pt minus 0pt}



\begin{document}



\begin{frontmatter}  %

\title{A Replication: A Reconsideration of Money Growth Rules}

% Set to FALSE if wanting to remove title (for submission)




\author[Add1]{Jacques Rossouw}
\ead{21159793@sun.ac.za}

\author[Add1]{Samantha Scott}
\ead{20945043@sun.ac.za}

\author[Add1,Add2]{Emma Terblanche}
\ead{21777039@sun.ac.za}



\address[Add1]{Stellenbosch University, Cape Town, South Africa}



\vspace{1cm}


\begin{keyword}
\footnotesize{
Macroeconomics \sep Calibration \sep Monetary Policy \\
\vspace{0.3cm}
}
\end{keyword}



\vspace{0.5cm}

\end{frontmatter}



%________________________
% Header and Footers
%%%%%%%%%%%%%%%%%%%%%%%%%%%%%%%%%
\pagestyle{fancy}
\chead{}
\rhead{}
\lfoot{}
\rfoot{\footnotesize Page \thepage}
\lhead{}
%\rfoot{\footnotesize Page \thepage } % "e.g. Page 2"
\cfoot{}

%\setlength\headheight{30pt}
%%%%%%%%%%%%%%%%%%%%%%%%%%%%%%%%%
%________________________

\headsep 35pt % So that header does not go over title




\newpage

\hypertarget{introduction}{%
\section{Introduction}\label{introduction}}

According to Belongia and Ireland (2020), nominal interest rate
management has been the Federal Reserve's approach to stabilise the
output gap and inflation. The Taylor rule is a good predictor of the
nominal interest rate management. Interest rate management has been
endorsed by literature and has shown advantages over the management of
money stock, however, economists have only argued against a constant
monetary growth rule but have neglected the potential benefits of a
flexible monetary growth rule. Belongia and Ireland (2020) is a
reconsideration of money growth rules in an estimated New Keynesian
model. This paper provides a replication of the New Keynesian model by
Belongia and Ireland (2020) as well as extending the authors' work via
calibration, as opposed to estimation. This paper examines the responses
of macroeconomic variables under the Taylor rule, Interest Rate rule,
Flexible Money Growth rate rule as well as the Constant Money Growth
rate rule, in response to the different shocks. The shocks are a
preference shock, a productivity shock, a money demand shock and a cost
push shock. The macroeconomic variables focused on are output growth
rates, inflation, nominal interest rates, money growth rates and output
gap.

In section 2, the model is explained. The first order conditions and the
log-linearisations of the model are then presented in sections 3 and 4,
respectively. Section 5 provides an explanation of the data and
calibration methods used. The paper presents and discusses the
calibration results, in the form of impulse response functions in
section 6. Lastly, this paper concludes by \ldots{}

\hypertarget{the-model}{%
\section{The Model}\label{the-model}}

The model economy from Belongia and Ireland (2020) includes the
representative household, a representative finished goods-producing
firm, ``i'' intermediate goods-producing firms (such that
\(i \in [0; 1]\), is a continuum) and a central bank. For each period, a
unique intermediate good is produced by each intermediate
goods-producing firm \(i\), such that the intermediate goods adopt the
same indexing notation, \(i \in [0; 1]\). The symmetry of this model
suggests that the focus can be narrowed to a representative intermediate
goods-producing firm instead. This representative firm then produces
good intermediate good \(i\).

The household preferences are described by their expected utility
function. These preferences, along with the incomplete indexation of
sticky nominal goods prices that are determined by the monopolistically
competitive intermediate goods-producing firms suggests a New Keynesian
IS and Phillips Curve that are both forward, and backwards-looking.
Monetary Policy is assumed to follow a version of the Taylor (1993)
rule, in line with the Federal Reserve behaviour over the sample period
from 1983 to 2019. To allow for alternative monetary policy rules, the
money demand curve used in the paper is chosen such that it may be
consistent with US data ranging over the same sample period. With the
basic model set up in this way, the derivations of the model's first
order conditions for each representative is replicated in the following
sections, followed by the complete system of linearised equations.

\hypertarget{derivations-of-models-first-order-conditions}{%
\section{Derivations of Model's First Order
Conditions:}\label{derivations-of-models-first-order-conditions}}

\hypertarget{the-representative-household}{%
\subsection{The Representative
Household}\label{the-representative-household}}

Each period \(t=0,1,2,...\) is entered by the representative household
with \(M_{t-1}\) units of money and \(B_{t-1}\) bonds. The household
receives a lump-sum monetary transfer \(T_{t}\) from the central bank at
the beginning of period \(t\). In addition, the household's bonds
mature, yielding \(B_{t-1}\) additional units of money. The household
uses some of this money to purchase \(B_{t}\) new bonds at the price of
\(1=rt\) units of money per bond. Therefore, \(r_t\) represents the
gross nominal interest rate between \(t\) and \(t + 1\). The household
provides \(h_{t}{(i)}\) units of labour to each intermediate
goods-producing firm \(i\in{[0; 1]}\) during period \(t\). The household
is paid at the nominal wage rate \(W_{t}\). Households therefore earn
\(W_{t}h_{t}\) in labour income, where

\[h_{t}=\int h_{t}{(i)} \;di\]

represents the total hours worked during the period. In period \(t\),
the household consumes \(C_{t}\) units of the finished goods. Finished
goods are purchased at the nominal price, \(P_{t}\) from the
representative finished goods-producing firm. The household receives
nominal profits \(D_{t}(i)\) from each intermediate goods-producing firm
\(i \in {[0; 1]}\), at the end of period \(t\). The household then
carries \(M_{t}\) units of money into period \(t + 1\), which is subject
to the following budget constraint:

\[\frac{M_{t-1}+T_{t}+B_{t-1}+W_{t}h_{t}+D_{t}}{P_{t}} \ge C_{t} + \frac{M_{t} + \frac{B_{t}}{r_{t}}}{P_{t}}\]

for all \(t=0,1,2,...,\) where

\[D_{t}= \int_{0}^{1}D_{t}{(i)}\;di\]

represents total profits received for the period. The household's
preferences are described by the following expected utility function:

\[E_{0}\sum_{t=0}^{\infty}\beta^{t}a_{t}[ln(C_{t} - \gamma C_{t-1}) + v(\frac{M_{t}}{P_{t}Z_{t}}, u_t)-(\frac{\phi_m}{2})(\frac{M_t/P_t}{zM_{t-1}/P_{t-1}}-1)^{2}(\frac{M_{t}}{P_{t}Z_{t}})-h_{t}]\]

Both the discount factor and the habit formation parameter lie between
zero and one, with \(0<\beta<1\) and \(0 \ge \gamma \le1\). The
preference shock \(a_{t}\) follows the stationary autoregressive process
below:

\[\ln(a_{t})=\rho_{a} \ln(a_{t-1}) + \varepsilon_{at} \tag{2}\]

for all \(t=0,1,2,...,\) with \(0 \le \rho_{a} \le 1\), where the
serially uncorrelated innovation \(\varepsilon_{at}\) is normally
distributed with a mean of zero and standard deviation \(\sigma_{a}\).
Total utility is a combination of the utility obtained from consumption,
real balances, and hours worked, which implies additive seperability, so
as to imply a specification for the model's IS curve that does not
include terms involving money and employment. As shown by King, Plosser,
and Rebelo (1988), additive separability also implies that a logarithmic
specification over consumption is necessary for the model to be
consistent with balanced growth. Also for balanced growth, real balances
\(M_{t}=P_{t}\) enter utility through the function \(v\) after being
scaled by the aggregate productivity shock \(Z_{t}\). This follows a
random walk with drift:

\[\ln(Z_{t})= \ln{z} + \ln(Z_{t-1}) + \varepsilon_{zt} \tag{3}\]

where the serially uncorrelated error term or innovation is normally
distributed with zero mean and standard deviation \(\sigma_{z}\). The
shock \(u_{t}\) to money demand follows the stationary autoregressive
process:

\[\ln(u_{t})= \rho_{u} \ln(u_{t-1}) + \varepsilon_{ut} \tag{4}\] for all
\(t=0,1,2,...\) with \(0 \le \rho_{u} <1\), where the serially
uncorrelated innovation \(\varepsilon_{ut}\) is normally distributed
with mean zero and standard deviation \(\sigma_{u}\). Finally, the
parameter \(\phi_{m} \ge 0\) governs the magnitude of the adjustment
cost for real balances, adapted from Nelson (2002) and Andres,
Lopez-Salido, and Nelson (2004, 2009) to take the quadratic functional
form used here. Since these costs subtract from utility along with hours
worked, they have the interpretation as a time cost, and are scaled by
the average growth rate parameter \(z\) from (3) so as to equal zero in
the model's steady state. Thus, the household chooses \(Ct, ht, Bt\),
and \(Mt\) for all \(t = 0, 1, 2,...,\) to maximize expected utility
subject to the budget constraint (1) for all \(t = 0,1,2,...,\)

The Household preferences are characterised by the following expected
utility function:

\[ EU(\cdot_t) = E_{0} \sum_{t=0}^\infty \beta^ta_t U(\cdot_t) = E_{0} \sum_{t=0}^\infty \beta^ta_t[\ln(C_{t}-\gamma C_{t-1})+v(\frac{M_t}{P_tZ_t},u_t)-\frac{\phi_m}{2}(\frac{\frac{M_t}{P_t}}{\frac{zM_{t-1}}{P_{t-1}}}-1)^2(\frac{M_t}{P_tZ_t})-h_t]\]

Given the Household budget constraint:

\[ \left[ \frac{M_{t-1} + T_t +B_{t-1} + W_th_t+D_t}{P_t} \ge C_t + \frac{M_t+\frac{B_t}{r_t}}{P_t} \right]\]

Combining the expected utility and budget constraint above, we can write
the LaGrangian for the Household:

\[  \mathcal{L}(\cdot_t) = a_tU( \cdot_{t}) + \beta E_t \mathcal{L}(\cdot_{t+1}) + \Lambda_t \left[ \frac{M_{t-1} + T_t +B_{t-1} + W_th_t+D_t}{P_t}-C_t - \left(\frac{M_t+\frac{B_t}{r_t}}{P_t} \right) \right] \]

where:

\[\begin{aligned} \mathcal{L}(\cdot_{t}) = \mathcal{L}(C_t, h_t, B_t, M_t) = &E_{0} \sum_{t=0}^\infty \beta^ta_t \left[\ln(C_{t}-\gamma C_{t-1})+v(\frac{M_t}{P_tZ_t},u_t)-\frac{\phi_m}{2}(\frac{\frac{M_t}{P_t}}{\frac{zM_{t-1}}{P_{t-1}}}-1)^2(\frac{M_t}{P_tZ_t})-h_t \right]\\
& + E_{0} \sum_{t=0}^\infty\left[ \Lambda_t \left( \frac{M_{t-1} + T_t +B_{t-1} + W_th_t+D_t}{P_t}-C_t - \left(\frac{M_t+\frac{B_t}{r_t}}{P_t} \right) \right) \right] \end{aligned}\]

\hypertarget{the-foc-with-respect-to-c_t}{%
\subsubsection{\texorpdfstring{The FOC with respect to
\(C_t\):}{The FOC with respect to C\_t:}}\label{the-foc-with-respect-to-c_t}}

\[\frac{\partial \mathcal{L}(C_t, h_t, B_t, M_t)}{\partial C_t}= a_t(\frac{1}{C_t-\gamma C_{t-1}})-\Lambda_t + E_t\beta a_{t+1}(-\gamma)(\frac{1}{C_{t+1}-\gamma C_t})\]

Equate to zero and solve:

\[0=  a_t(\frac{1}{C_t-\gamma C_{t-1}})-\Lambda_t + E_t\beta a_{t+1}(-\gamma)(\frac{1}{C_{t+1}-\gamma C_t})\]

\[\Lambda_t =  a_t(\frac{1}{C_t-\gamma C_{t-1}}) + E_t\beta a_{t+1}(-\gamma)(\frac{1}{C_{t+1}-\gamma C_t}) \tag{5}\]

\hypertarget{the-foc-with-respect-to-h_t}{%
\subsubsection{\texorpdfstring{The FOC with respect to
\(h_t\):}{The FOC with respect to h\_t:}}\label{the-foc-with-respect-to-h_t}}

\[\frac{\partial \mathcal{L}(C_t, h_t, B_t, M_t)}{\partial h_t}= (-1)( a_t) + (\frac{W_t}{P_t})(\Lambda_t)\]

Equate to zero and solve:

\[0 = (-1)( a_t) + (\frac{W_t}{P_t})(\Lambda_t)\]
\[( a_t) = (\frac{W_t}{P_t})(\Lambda_t) \]
\[ a_t = (\frac{W_t}{P_t})(\Lambda_t) \tag{6}\]

\hypertarget{the-foc-with-respect-to-b_t}{%
\subsubsection{\texorpdfstring{The FOC with respect to
\(B_t\):}{The FOC with respect to B\_t:}}\label{the-foc-with-respect-to-b_t}}

\[\frac{\partial L(C_t, h_t, B_t, M_t)}{\partial B_t}= (\frac{-1}{r_t P_t})(\Lambda_t) + E_t\beta \Lambda_{t+1}(\frac{1}{P_{t+1}})\]

Equate to zero and solve:

\[ 0=(\frac{-1}{r_t P_t})(\Lambda_t) + E_t\beta \Lambda_{t+1}(\frac{1}{P_{t+1}})\]
\[ (\frac{1}{r_t P_t})(\Lambda_t) = E_t\beta \Lambda_{t+1}(\frac{1}{P_{t+1}})\]
\[ \Lambda_t = (r_t P_t) E_t\beta\Lambda_{t+1}(\frac{1}{P_{t+1}})\]
\[ \Lambda_t = \beta(r_t) E_t(\frac{P_t\Lambda_{t+1}}{P_{t+1}})\]

Because: \[\pi_t=\frac{P_t}{P_{t-1}}\]

Therefore: \[\frac{P_t}{P_{t+1}}=\frac{1}{\pi_{t+1}}\]

As such, the equation can be rewritten as:

\[\Lambda_t = \beta(r_t)E_t(\frac{\Lambda_{t+1}}{\pi_{t+1}}) \tag{7}\]

\hypertarget{the-foc-with-respect-to-m_t}{%
\subsubsection{\texorpdfstring{The FOC with respect to
\(M_t\):}{The FOC with respect to M\_t:}}\label{the-foc-with-respect-to-m_t}}

\[\frac{\partial L(C_t, h_t, B_t, M_t)}{\partial M_t}=a_t[v_1(\frac{M_t}{P_tZ_t}, u_t) \frac{1}{P_t Z_t}-\phi_m(\frac{\frac{M_t}{P_t}}{\frac{z_tM_{t-1}}{P_{t-1}}}-1)(\frac{\frac{1}{P_{t}}}{\frac{ZM_{t-1}}{P_{t-1}}})(\frac{M_t}{P_tZ_t})-\frac{\phi_m}{2}(\frac{\frac{M_t}{P_t}}{\frac{z_tM_{t-1}}{P_{t-1}}}-1)^2(\frac{1}{P_tZ_t})]\]
\[ + \Lambda_t[-\frac{1}{P_t}] + E_t\beta a_{t+1} \left[-\phi_m(\frac{M_{t+1}/P_{t+1}}{ZM_t/P_t}-1)(-\frac{M_{t+1}/P_{t+1}}{zM_{t}^2/P_{t}})(\frac{M_{t+1}}{P_{t+1}Z_{t+1}}) \right]+E_t \beta \Lambda_{t+1}(\frac{1}{P_{t+1}})\]

where:

\[ v_1(\cdot) = \frac{\partial}{\partial M_t} v(\cdot) \]

We can now use equation (7) and the fact that
\(\frac{P_t}{P_{t-1}} = \pi_t\), to yield:

\[ \beta E_t \Lambda_{t+1} \frac{1}{P_{t+1}} = \frac{1}{P_tr_t} \Lambda_t \]

multiply by \(\frac{1} {P_t Z_t}\) throughout and set equal to zero:

\[\begin{aligned} \therefore 0 = & a_tv_1(\frac{M_t}{P_tZ_t}, u_t) - a_t\frac{\phi_m}{2}(\frac{\frac{M_t}{P_t}}{\frac{zM_{t-1}}{P_{t-1}}}-1)^2 -  a_t \phi_m(\frac{\frac{M_t}{P_t}}{\frac{zM_{t-1}}{P_{t-1}}}-1)(\frac{\frac{M_t}{P_{t}}}{\frac{zM_{t-1}}{P_{t-1}}})\\
& - \Lambda_t Z_t + \frac{Z_t}{r_t} \Lambda_t + E_t\beta a_{t+1} \left[\phi_m(\frac{M_{t+1}/P_{t+1}}{ZM_t/P_t}-1) \left(\frac{M_{t+1}/P_{t+1}}{zM_{t}/P_{t}} \right)^2(\frac{zZ_t}{Z_{t+1}}) \right] \end{aligned}\]

Rearranging yields the final result:

\[\begin{aligned}
&a_{t} v_{1}\left(\frac{M_{t}}{P_{t} Z_{t}}, u_{t}\right)-a_{t}\left(\frac{\phi_{m}}{2}\right)\left(\frac{M_{t} / P_{t}}{z M_{t-1} / P_{t-1}}-1\right)^{2} \\
&-a_{t} \phi_{m}\left(\frac{M_{t} / P_{t}}{z M_{t-1} / P_{t-1}}-1\right)\left(\frac{M_{t} / P_{t}}{z M_{t-1} / P_{t-1}}\right) \\
&+\beta \phi_{m} E_{t}\left[a_{t+1}\left(\frac{M_{t+1} / P_{t+1}}{z M_{t} / P_{t}}-1\right)\left(\frac{M_{t+1} / P_{t+1}}{z M_{t} / P_{t}}\right)^{2}\left(\frac{z Z_{t}}{Z_{t+1}}\right)\right] \\
&=Z_{t} \Lambda_{t}\left(1-\frac{1}{r_{t}}\right)
\end{aligned} \tag{8}\]

Using the fact that:

\[v_1(\frac{M_t}{P_tZ_t},u_t) = \frac{1}{\delta}[\ln(m*) - \ln(\frac{M_t}{P_tZ_t}) + \ln(u_t)]\]

We can rewrite (8) to yield:

\[\begin{array}{l}
\frac{a_{t}}{\delta}\left[\ln \left(m^{*}\right)-\ln \left(\frac{M_{t}}{P_{t} Z_{t}}\right)+\ln \left(u_{t}\right)\right]-a_{t}\left(\frac{\phi_{m}}{2}\right)\left(\frac{M_{t} / P_{t}}{z M_{t-1} / P_{t-1}}-1\right)^{2} \\
-a_{t} \phi_{m}\left(\frac{M_{t} / P_{t}}{z M_{t-1} / P_{t-1}}-1\right)\left(\frac{M_{t} / P_{t}}{z M_{t-1} / P_{t-1}}\right) \\
+\beta \phi_{m} E_{t}\left[a_{t+1}\left(\frac{M_{t+1} / P_{t+1}}{z M_{t} / P_{t}}-1\right)\left(\frac{M_{t+1} / P_{t+1}}{z M_{t} / P_{t}}\right)^{2}\left(\frac{z Z_{t}}{Z_{t+1}}\right)\right] \\
=Z_{t} \Lambda_{t}\left(1-\frac{1}{r_{t}}\right)
\end{array} \tag{9}\]

\hypertarget{the-representative-finished-goods-producing-firm}{%
\subsection{The Representative Finished Goods-Producing
Firm}\label{the-representative-finished-goods-producing-firm}}

The representative finished goods-producing firm uses \(Y_t(i)\) units
of each intermediate good \(i \in [0,1]\). These intermediate goods are
bought at the nominal price \(P_t(i)\) in order to produce \(Y_t\) units
of the final good according to the technology described by

\[[\int_0^1Y_t(i)^{\frac{(\theta_t-1)}{\theta_t}} di]^{\frac{\theta_t}{\theta_t-1}} \ge Y_t\]

where \(\theta_t\) translates into a random shock to the intermediate
goods-producing firms' desired markup of price over marginal cost and
therefore acts like a cost push shock of the kind introduced into the
New Keynesian model by Clarida, Gali, and and Gertler (1999). Here, this
markup shock follows the stationary autoregressive process

\[\ln(\theta_t)=(1-\rho_{\theta})\ln(\theta)+\rho_{\theta}\ln(\theta_{t-1})+\varepsilon_{\theta t} \tag{10}\]

for all \(t=0,1,2,...,\), where the serially uncorrelated innovation
\(\varepsilon_{\theta t}\) is normally distributed with mean zero and
standard deviation \(\sigma_{\theta}\). Thus, during each period \(t\),
the finished good-producing firm chooses \(Y_t(i)\) for all
\(i \in [0,1]\) to maximize its profits, which are given by

\[P_t[\int_0^1Y_t(i)^{\frac{\theta_t-1}{\theta_t}} di]^{\frac{\theta_t}{\theta_t-1}} - \int_0^1P_t(i)Y_t(i) \;di\]

The Langrangian function from which the first order conditions are
derived, is as follows:

\[L = \int_0^1P_t(i)Y_t(i) \; di +\lambda_t[Y_t - (\int_0^1Y_t(i)^{\frac{\theta_t-1}{\theta_t}} \;di)^{\frac{\theta_t}{\theta_t-1}}]\]

\hypertarget{the-foc-with-respect-to-y_ti}{%
\subsubsection{\texorpdfstring{The FOC with respect to
\(Y_t(i)\):}{The FOC with respect to Y\_t(i):}}\label{the-foc-with-respect-to-y_ti}}

\[\frac{\partial L}{\partial Y_t(i)} = P_t(i) - \lambda_t[{\frac{\theta_t}{\theta_t-1}} (\int_0^1 Y_t(i)^{\frac{\theta_t-1}{\theta_t}} di)^{\frac{\theta_t}{\theta_t-1} -1} \frac{\theta_t-1}{\theta_t} Y_t(i)^{\frac{\theta_t-1}{\theta_t} -1}] = 0\]
\[P_t(i) - \lambda_t[{\frac{\theta_t}{\theta_t-1}} (\int_0^1 Y_t(i)^{\frac{\theta_t-1}{\theta_t}} di)^{\frac{1}{\theta_t-1}} \frac{\theta_t-1}{\theta_t} Y_t(i)^{\frac{-1}{\theta_t}}] = 0\]

\[ \lambda_t[\int_0^1 Y_t(i)^{\frac{\theta_t-1}{\theta_t}} di]^{\frac{1}{\theta_t-1}} Y_t(i)^{\frac{-1}{\theta_t}} = P_t(i)\]

\[ \lambda_t[\int_0^1 Y_t(i)^{\frac{\theta_t-1}{\theta_t}} di]^{\frac{1}{\theta_t-1}}  = P_t(i)Y_t(i)^{\frac{1}{\theta_t}}\]
\[ \lambda_t[\int_0^1 Y_t(i)^{\frac{\theta_t-1}{\theta_t}} di]^{\frac{1}{\theta_t-1}}P_t(i)^{-1}  = Y_t(i)^{\frac{1}{\theta_t}}\]

\[ \lambda_t^{\theta_t} [\int_0^1 Y_t(i)^{\frac{\theta_t-1}{\theta_t}} di]^{\frac{\theta_t}{\theta_t-1}}P_t(i)^{-\theta_t}  = Y_t(i)\]

\[ \lambda_t^{\theta_t} Y_t P_t(i)^{-\theta_t} = Y_t(i)\] Where
\(Y_t = [\int_0^1 Y_t(i)^{\frac{\theta_t-1}{\theta_t}} di]^{\frac{\theta_t}{\theta_t-1}}\)

\[ (\frac{\lambda_t}{P_t(i)})^{\theta_t} Y_t  = Y_t(i)\] Sub
\(Y_t(i) = (\frac{\lambda_t}{P_t(i)})^{\theta_t} Y_t\) into \(Y_t\)

\[[\int_0^1 ((\frac{\lambda_t}{P_t(i)})^{\theta_t} Y_t)^{\frac{\theta_t-1}{\theta_t}} di]^{\frac{\theta_t}{\theta_t-1}}  = Y_t\]
\[[\int_0^1 ((\frac{P_t(i)}{\lambda_t})^{-\theta_t})^{\frac{\theta_t-1}{\theta_t}} di]^{\frac{\theta_t}{\theta_t-1}}  = 1\]
\[(\frac{1}{\lambda_t})^{-\theta_t}[\int_0^1 (P_t(i))^{-\theta_t+1} di]^{\frac{\theta_t}{\theta_t-1}}  = 1\]

\[[\int_0^1 (P_t(i))^{1-\theta_t} di]^{\frac{\theta_t}{\theta_t-1}}  = {\lambda_t}^{-\theta_t}\]
\[[\int_0^1 (P_t(i))^{1-\theta_t} di]^{\frac{1}{1-\theta_t}}  = {\lambda_t}\]
Therefore:

\[\lambda_t = P_t\] Sub \(Y_t = P_t\) back in:

\[(\frac{P_t}{P_t(i)})^{\theta_t} Y_t = Y_t(i)\]

Therefore:

\[Y_t(i) = (\frac{P_t(i)}{P_t})^{-\theta_t} Y_t\]

\hypertarget{the-representative-intermediate-goods-producing-firm}{%
\subsection{The Representative Intermediate Goods-Producing
Firm}\label{the-representative-intermediate-goods-producing-firm}}

During each period \(t=0,1,2,...,\) the representative intermediate
goods-producing firm hires \(h_t(i)\) units of labour from the
representative household to manufacture \(Y_t(i)\) units of intermediate
good \(i\) according to the technology described by:

\[Z_th_t(i) \ge Y_t(i) \tag{11}\]

where \(Z_t\) is the aggregate productivity shock introduced in (3).

The quadratic cost of adjusting its nominal price between periods is
given by:

\[\frac{\phi_p}{2}[\frac{P_t(i)}{{\pi_{t-1}^a \pi^{1-a}P_{t-1}(i)}}-1]^2Y_t\]

The firm chooses \(P_t(i)\) for all \(t = 0,1,2,...\) to maximise its
total real market value proportional to:

\[E_0\sum_{t=0}^\infty \beta^t\Lambda_t[\frac{D_t(i)}{P_t}]\]

where the firm's real profits are measured by:

\[\frac{D_t(i)}{P_t}=[\frac{P_t(i)}{P_t}]^{1-\theta_t}Y_t-[\frac{P_t(i)}{P_t}]^{-\theta_t}(\frac{W_t}{P_t})(\frac{Y_t}{Z_t})-\frac{\phi}{2}[\frac{P_t(i)}{\pi_{t-1}^a \pi^{1-a}P_{t-1}(i)}]^2Y_t\tag{12}\]

The Value function of this optimisation problem is:
\[v=maxE_t\sum_{t=0}^\infty\beta^t\Lambda_t{\frac{D_t(i)}{P_t}}\]

As such, the Bellman Equation can be constructed as:

\[v=\beta^t\Lambda_t{\frac{D_t(i)}{P_t}} + \beta^{t+1}E_t\Lambda_{t+1}{\frac{D_{t+1}(i)}{P_{t+1}}}\]
\[v=\beta^t\Lambda_t[(\frac{P_t(i)}{P_t})^{1-\theta_t}Y_t-(\frac{P_t(i)}{P_t})^{-\theta_t}(\frac{W_t}{P_t})(\frac{Y_t}{Z_t})-\frac{\phi}{2}(\frac{P_t(i)}{\pi_{t-1}^a \pi^{1-a}P_{t-1}(i)})^2Y_t] + \beta^{t+1}E_t\Lambda_{t+1}[(\frac{P_{t+1}(i)}{P_{t+1}})^{1-\theta_{t+1}}Y_{t+1}\]

\[-(\frac{P_{t+1}(i)}{P_{t+1}}]^{-\theta_{t+1}}(\frac{W_{t+1}}{P_{t+1}})(\frac{Y_{t+1}}{Z_{t+1}})-\frac{\phi_p}{2}[\frac{P_{t+1}(i)}{\pi_{t}^a \pi^{1-a}P_{t}(i)}]^2Y_{t+1}]\]

\hypertarget{the-foc-with-respect-to-p_ti}{%
\subsubsection{\texorpdfstring{The FOC with respect to
\(P_t(i)\)}{The FOC with respect to P\_t(i)}}\label{the-foc-with-respect-to-p_ti}}

\[ \frac{\partial v}{\partial P_t(i)} = \beta^t\Lambda_t(1-\theta_t)(\frac{P_t(i)}{P_t})^{-\theta_t}(\frac{1}{P_t})Y_t-\beta^t\Lambda_t[(-\theta_t\frac{P_t(i)}{P_t})^{-\theta_t-1}(\frac{1}{P_t})(\frac{W_t}{P_t})(\frac{Y_t}{Z_t})]\]
\[-\beta^t\Lambda_t[\phi_p(\frac{P_t(i)}{\pi_{t-1}^a \pi^{1-a}P_{t-1}(i)}-1)(\frac{1}{\pi_{t-1}^a \pi^{1-a}P_{t-1}(i)})Y_t] -\beta^{t+1}E_t\Lambda_{t+1}\phi_p[(\frac{P_{t+1}(i)}{\pi_t^a\pi^{1-a}P_{t}(i)}-1)(\frac{-P_{t+1}(i)Y_{t+1}}{\pi_t^a\pi^{1-a}P_{t}(i)^2})] =0\]

Next, in order to cancel out most \(\Lambda_t\), \(Y_t\) and \(P_t\)'s,
multiply through by \(\frac{P_t}{\Lambda_t Y_t}\):

\[\beta^t(1-\theta_t)(\frac{P_t(i)}{P_t})^{-\theta_t} + \beta^t[\theta_t(\frac{P_t(i)}{P_t})^{-\theta_t-1}(\frac{W_t}{P_t})(\frac{1}{Z_t})] - \beta^t\phi_p[(\frac{P_t(i)}{\pi_{t-1}^a \pi^{1-a}P_{t-1}(i)}-1)(\frac{P_t}{\pi_{t-1}^a \pi^{1-a}P_{t-1}(i)})]\]
\[  +\beta^{t+1}\phi_pE_t(\frac{\Lambda_{t+1}}{\Lambda_{t}})(\frac{P_{t+1}(i)}{\pi_t^a\pi^{1-a}P_{t}(i)}-1)(\frac{P_{t+1}(i)}{\pi_t^a\pi^{1-a}P_{t}(i)})(\frac{Y_{t+1}}{Y_t})(\frac{P_t}{P_t(i)})=0\]

\[(1-\theta_t)(\frac{P_t(i)}{P_t})^{-\theta_t} + \theta_t(\frac{P_t(i)}{P_t})^{-\theta_t-1}(\frac{W_t}{P_t})(\frac{1}{Z_t}) - \phi_p[(\frac{P_t(i)}{\pi_{t-1}^a \pi^{1-a}P_{t-1}(i)}-1)(\frac{P_t}{\pi_{t-1}^a \pi^{1-a}P_{t-1}(i)})]\]
\[ + \beta\phi_pE_t(\frac{\Lambda_{t+1}}{\Lambda_{t}})(\frac{P_{t+1}(i)}{\pi_t^a\pi^{1-a}P_{t}(i)}-1)(\frac{P_{t+1}(i)}{\pi_t^a\pi^{1-a}P_{t}(i)})(\frac{Y_{t+1}}{Y_t})(\frac{P_t}{P_t(i)})=0 \tag{13}\]

and (11) with equality for all \(t=0,1,2,...,\)

\hypertarget{the-efficient-level-of-output-and-output-gap}{%
\subsection{The Efficient Level of Output and Output
Gap}\label{the-efficient-level-of-output-and-output-gap}}

A social planner for this economy who can overcome the frictions
associated with monetary trade, sluggish price adjustment, and the
monopolistically competitive structure of the intermediate
goods-producing sector chooses \(Q_t\) and \(n_t(i)\) for all
\(i \in [0,1]\) to maximize the social welfare function

\[E_0\sum_{t=0}^{\infty}\beta^ta_t[\ln(Q_t-\gamma Q_{t-1})-\int_0^1n_t(i) di]\]
subject to

\[Z_t[\int_0^1n_t(i) di]^{\frac{\theta_t}{\theta_t-1}} \ge Q_t\] As
such, the Bellman Equation of this model is:

\[v(Q_{t-1}) = a_t[\ln(Q_t-\gamma Q_{t-1})-\int_0^1n_t(i) di] + \beta E_tv(Q_t,Q_{t+1}) + \Xi [Z_t(\int_0^1n_t(i) di)^{\frac{\theta_t}{\theta_t-1}} - Q_t]\]

\hypertarget{the-foc-with-respect-to-q_t}{%
\subsubsection{\texorpdfstring{The FOC with respect to
\(Q_t\):}{The FOC with respect to Q\_t:}}\label{the-foc-with-respect-to-q_t}}

\[\frac{\partial v(Q_{t-1})}{\partial Q_t} = \frac{a_t}{Q_t-\gamma Q_{t-1}} + \beta E_t\frac{\partial v(Q_t, Q_{t+1})}{\partial Q_t} - \Xi =0\]
We know that

\[\frac{\partial v(Q_{t-1})}{\partial Q_{t-1}} = \frac{a_t}{Q_t-\gamma Q_{t-1}} (-\gamma)\]

Therefore, when iterating forward, we find that:
\[\frac{\partial v(Q_{t})}{\partial Q_{t}} = \frac{a_{t+1}}{Q_{t+1}-\gamma Q_{t}} (-\gamma)\]

Subbing \(\frac{\partial v(Q_{t})}{\partial Q_{t}}\) back in gives:

\[\frac{\partial v(Q_{t-1})}{\partial Q_t} = \frac{a_t}{Q_t-\gamma Q_{t-1}} + \beta E_t\frac{a_{t+1}}{Q_{t+1}-\gamma Q_{t}} (-\gamma) - \Xi =0\]
Multiply through by \((-1)\):

\[ \frac{a_t}{Q_t-\gamma Q_{t-1}} - \beta E_t\frac{a_{t+1}}{Q_{t+1}-\gamma Q_{t}} (\gamma)  = \Xi\]

\hypertarget{the-foc-with-respect-to-n_t}{%
\subsubsection{\texorpdfstring{The FOC with respect to
\(n_t\):}{The FOC with respect to n\_t:}}\label{the-foc-with-respect-to-n_t}}

\[\frac{\partial v(Q_{t-1})}{\partial n_t} = a_t (-1) + \Xi Z_t \frac{\theta_t}{\theta_t -1} (\int_0^1n_t(i)^\frac{\theta_t -1}{\theta_t} di)^\frac{\theta_t}{\theta_t - 1} (\frac{\theta_t -1}{\theta_t}) n_t(i)^{\frac{\theta_t -1}{\theta_t}-1}=0\]
\[ a_t + \Xi Z_t  (\int_0^1n_t(i)^\frac{\theta_t -1}{\theta_t} di)^\frac{1}{\theta_t - 1} n_t(i)^{\frac{-1}{\theta_t}}=0\]
\[  \Xi Z_t  (\int_0^1n_t(i)^\frac{\theta_t -1}{\theta_t} di)^\frac{1}{\theta_t - 1} n_t(i)^{\frac{-1}{\theta_t}}=a_t\]

The Feasibility Constraint of this problem is:
\[(\int_0^1n_t(i)^\frac{\theta_t -1}{\theta_t} di)^\frac{\theta_t}{\theta_t - 1} \ge \frac{Q_t}{Z_t}\]
Rearranging the Feasibility Constraint gives:

\[(\int_0^1n_t(i)^\frac{\theta_t -1}{\theta_t} di)^\frac{1}{\theta_t - 1} \ge (\frac{Q_t}{Z_t})^\frac{1}{\theta_t}\]
Therefore

\[\Xi Z_t (\frac{Q_t}{Z_t})^\frac{1}{\theta_t}n_t(i)^{\frac{-1}{\theta_t}}=a_t\]

\newpage

\hypertarget{log-linearisation}{%
\section{Log-Linearisation:}\label{log-linearisation}}

\hypertarget{obtaining-the-final-system-of-equations}{%
\subsection{Obtaining the final system of
equations:}\label{obtaining-the-final-system-of-equations}}

\hypertarget{conditions}{%
\subsubsection{Conditions:}\label{conditions}}

To rewrite the equations into a usable system of equations that can be
log-linearised, we make use of the following series of steady state,
stationary or equilibrium equations and variables:

\hypertarget{a}{%
\paragraph{A:}\label{a}}

\[ Y_{t}(i)=Y_{t}, h_{t}(i)=h_{t}, D_{t}(i)=D_{t}, \text { and } P_{t}(i)=P_{t} \text { for all } i \in[0,1] \text { and } t=0,1,2, \ldots \]

\hypertarget{b}{%
\paragraph{B:}\label{b}}

\[ M_{t}=M_{t-1}+T_{t} \text { and } B_{t}=B_{t-1}=0 \]

\hypertarget{c}{%
\paragraph{C:}\label{c}}

\[ y_{t}=Y_{t} / Z_{t}, c_{t}=C_{t} / Z_{t}, m_{t}=\left(M_{t} / P_{t}\right) / Z_{t}, q_{t}=Q_{t} / Z_{t}, \lambda_{t}=Z_{t} \Lambda_{t}, \text { and } z_{t}=Z_{t} / Z_{t-1} \]

\hypertarget{d}{%
\paragraph{D:}\label{d}}

In steady state we can write the following, per their definitions:

\[\begin{aligned} & y_{t}=y,\ c_{t}=c,\ \pi_{t}=\pi,\ r_{t}=r, \ m_{t}=m,\ q_{t}=q,\ x_{t}=x,\ \\ 
& \mu_{t}=\mu,\ g_{t}=g,\ \lambda_{t}=\lambda,\ a_{t}=a=1,\ z_{t}=z,\ u_{t}=u=1, \\ 
& \text {and } \theta_{t}=\theta \end{aligned}\]

\hypertarget{imposing-the-coniditions}{%
\subsubsection{Imposing The
Coniditions:}\label{imposing-the-coniditions}}

Firstly, we combine equation (11) and (12) from Section 3.3
\protect\hyperlink{the-representative-intermediate-goods-producing-firm}{The
Representative Intermediate Goods-Producing Firm} and impose the above
conditions to obtain equation (1) from the appendix of ``Belongia and
Ireland (2020)'':

\[Y_{t} = C_t + \frac{\phi_{p}}{2} \cdot \left[ \frac{\pi_{t}}{{\pi_{t-1}}^{\alpha} \pi^{1-\alpha}} -1 \right]^{2} \cdot Y_t \]

Dividing both sides by \(Z_t\) we can substitute for the necessary
stationarity variables and rewrite as:

\[y_{t}=c_{t}+\frac{\phi_{p}}{2}\left(\frac{\pi_{t}}{\pi_{t-1}^{\alpha} \pi^{1-\alpha}}-1\right)^{2} y_{t} \tag{1}\]

Directly from Equations (2)-(4) in Section 3.1
\protect\hyperlink{the-representative-household}{The Representative
Household}:

\[\ln \left(a_{t}\right)=\rho_{a} \ln \left(a_{t-1}\right)+\varepsilon_{a t} \tag{2}\]

\[\ln \left(Z_{t}\right)=\ln (z) + \ln(Z_{t-1}) +\varepsilon_{z t}\]

Subtract both sides by \(\ln(Z_{t-1})\)

\[\ln \left(z_{t}\right)=\ln (z)+\varepsilon_{z t} \tag{3}\]

\[\ln \left(u_{t}\right)=\rho_{u} \ln \left(u_{t-1}\right)+\varepsilon_{u t} \tag{4} \]

From \protect\hyperlink{the-foc-with-respect-to-c_t}{The FOC with
respect to \(C_t\):}

\[ \Lambda_t = \frac{a_t}{C_t-\gamma C_{t-1}} - \beta \gamma E_t \left[ \frac{a_{t+1}}{C_{t+1} - \gamma C_t} \right] \]

\[\Rightarrow \lambda_{t}=\frac{a_{t}z_{t}}{z_{t} c_{t}-\gamma c_{t-1}}-\beta \gamma E_{t}\left(\frac{a_{t+1}}{z_{t+1} c_{t+1}-\gamma c_{t}}\right) \tag{5}\]

\[\lambda_{t}=\beta r_{t} E_{t}\left(\frac{\lambda_{t+1}}{z_{t+1} \pi_{t+1}}\right) \tag{7}\]

Rewrite equation (9) from
\protect\hyperlink{the-foc-with-respect-to-m_t}{The FOC with respect to
\(M_t\):}

\[\begin{aligned}
&a_{t} v_{1}\left(\frac{M_{t}}{P_{t} Z_{t}}, u_{t}\right)-a_{t}\left(\frac{\phi_{m}}{2}\right)\left(\frac{M_{t} / P_{t}}{z M_{t-1} / P_{t-1}}-1\right)^{2} \\
&-a_{t} \phi_{m}\left(\frac{M_{t} / P_{t}}{z M_{t-1} / P_{t-1}}-1\right)\left(\frac{M_{t} / P_{t}}{z M_{t-1} / P_{t-1}}\right) \\
&+\beta \phi_{m} E_{t}\left[a_{t+1}\left(\frac{M_{t+1} / P_{t+1}}{z M_{t} / P_{t}}-1\right)\left(\frac{M_{t+1} / P_{t+1}}{z M_{t} / P_{t}}\right)^{2}\left(\frac{z Z_{t}}{Z_{t+1}}\right)\right] \\
&=Z_{t} \Lambda_{t}\left(1-\frac{1}{r_{t}}\right)
\end{aligned}\]

\[\begin{aligned} & \frac{a_{t}}{\delta}\left[\ln \left(m^{*}\right)-\ln \left(m_{t}\right)+\ln \left(u_{t}\right)\right]-a_{t}\left(\frac{\phi_{m}}{2}\right)\left(\frac{z_{t} m_{t}}{z m_{t-1}}-1\right)^{2} \\
& -a_{t} \phi_{m}\left(\frac{z_{t} m_{t}}{z m_{t-1}}-1\right)\left(\frac{z_{t} m_{t}}{z m_{t-1}}\right) \\
& +\beta \phi_{m} E_{t}\left[a_{t+1}\left(\frac{z_{t+1} m_{t+1}}{z m_{t}}-1\right)\left(\frac{z_{t+1} m_{t+1}}{z m_{t}}\right)^{2}\left(\frac{z}{z_{t+1}}\right)\right] \\
& =\lambda_{t}\left(1-\frac{1}{r_{t}}\right), \\
\end{aligned}\]

\hypertarget{applying-the-taylor-method-of-linear-approximation}{%
\subsection{Applying the Taylor Method of Linear
Approximation}\label{applying-the-taylor-method-of-linear-approximation}}

We apply the Taylor throughout the following section as follows:

For a function \(f(x_t)\) input \(x_t\) , we apply the Taylor method
such that: \[f(x_t) = f(X^{SS}) + \frac{df(x)}{x_t} (x_t(i) - x^{SS})\]

Using the approximation: \[\begin{aligned}
&x_t - x^{SS} \approx x^{SS} \hat{x}_{t}
\\
&\text{where } \hat{x}_{t} = \ln(x_t)-\ln(x^{SS})
\\
&\text{For short, we write: } x^{SS} \equiv x
\\
&\text{(i.e. no time subscript)}
\end{aligned}\]

\hypertarget{uhligs-method-of-linearisation}{%
\subsection{Uhlig's Method of
Linearisation}\label{uhligs-method-of-linearisation}}

Applying Uhlig's method requires the following:

\newpage

For the function \(f(x_t)\) i.e., we use the fact that:

\begin{enumerate}
\def\labelenumi{\arabic{enumi}.}
\item
  \[\hat{x}_t = \ln(x_t) - \ln(x) \\
  \text{ where x } \equiv x^{SS}\]
\item
  Then, \(f(^{SS})\) is used to simplify \(f(x_t)\) where \(x_t\) is
  replaced accodring to \(x_t = x \cdot e^{\hat{x}_t}\)
\item
  Additionally, the approximation result
  \(e^{x_t+y_t} \approx 1+x_t+y_t\) is used for further simplification.
\item
  For the last step, all constants can be dropped.
\end{enumerate}

For the following sections, we show the steps to reach the final
log-linearised equations (19) to (31) from in Section 2.7 of ``THE
PAPER''.

\hypertarget{equation-1-from-the-appendix-of-belongia-and-ireland-2020}{%
\subsection{Equation (1) from the appendix of ``Belongia and Ireland
(2020)''}\label{equation-1-from-the-appendix-of-belongia-and-ireland-2020}}

Recall the results from equation (11), (12), and the Budget Constraint:

\[Z_th_t(i) \ge Y_t(i) \tag{11}\]
\[\frac{D_t(i)}{P_t}=[\frac{P_t(i)}{P_t}]^{1-\theta_t}Y_t-[\frac{P_t(i)}{P_t}]^{-\theta_t}(\frac{W_t}{P_t})(\frac{Y_t}{Z_t})-\frac{\phi}{2}[\frac{P_t(i)}{\pi_{t-1}^a \pi^{1-a}P_{t-1}(i)}]^2Y_t\tag{12}\]
\[\frac{M_{t-1}+T_{t}+B_{t-1}+W_{t} h_{t}+D_{t}}{P_{t}} \geq C_{t}+\frac{M_{t}+B_{t} / r_{t}}{P_{t}}\]

We can now apply the equilibrium conditions to obtain the results:

\[Z_th_t = Y_t \]
\[\begin{aligned} \frac{D_t}{P_t} & =[\frac{P_t}{P_t}]^{1-\theta_t}Y_t - [\frac{P_t}{P_t}]^{-\theta_t}(\frac{W_t}{P_t})(\frac{Y_t}{Z_t})-\frac{\phi}{2}[\frac{P_t}{\pi_{t-1}^a \pi^{1-a}P_{t-1}}]^2Y_t \\
& = Y_t - (\frac{W_t}{P_t})(\frac{Y_t}{Z_t}) - \frac{\phi}{2}[\frac{P_t}{\pi_{t-1}^a \pi^{1-a}P_{t-1}}]^2Y_t \end{aligned}\]
\[\begin{aligned} C_{t} &= \frac{M_{t-1}+T_{t}+\frac{W_tY_t}{Z_t}+D_{t} - M_{t-1} - T_t}{P_{t}}\\
& = \frac{W_tY_t}{Z_t P_t} + \frac{D_{t}}{P_t}
\end{aligned}\]

\[\begin{aligned} -Y_t  &= -\left[ \frac{D_t}{P_t} + (\frac{W_t}{P_t})(\frac{Y_t}{Z_t}) \right] - \frac{\phi}{2}[\frac{P_t}{\pi_{t-1}^a \pi^{1-a}P_{t-1}}]^2Y_t \\
\therefore Y_t & = C_t - \frac{\phi}{2}[\frac{P_t}{\pi_{t-1}^a \pi^{1-a}P_{t-1}}]^2Y_t
\end{aligned}\]

Finally, we can now rewrite \& linearise the final equation:

\[\begin{aligned}
y_{t} &= c_t + \frac{\phi_{p}}{2} \cdot \left[ \frac{\pi_{t}}{{\pi_{t-1}}^{\alpha} \pi^{1-\alpha}} -1 \right]^{2} \cdot y\\
&=c_{t} + \frac{\phi_{p}}{2} \cdot \left(\frac{\pi_{t}}{{\pi_{t-1}}^{\alpha} \pi^{1-\alpha}} \right)^2 \cdot y_{t} - \phi_{p} \cdot
\frac{\pi_{t}}{{\pi_{t-1}}^{\alpha} \pi^{1-\alpha}} \cdot y_{t} + \frac{\phi_{p}}{2} \cdot y_{t}\\
&= c_t + p_{1} - p_{2} + p_{3}
\end{aligned}\]

Now, we apply the Taylor method to each \(p_{i}\), to get:

\[\begin{aligned}
p_{1}^{SS}&=\frac{\phi_{p} \cdot y}{2}\\\left[\frac{dp_{1}}{dy_{t}} \right]^{SS} &= \frac{\phi_{p}}{2}\\\left[\frac{dp_{1}}{d\pi_{t}} \right]^{SS} &= \frac{\phi_{p} \cdot y}{\pi}\\\left[\frac{dp_{1}}{d\pi_{t-1}} \right]^{SS} &= - \frac{\phi_{p} \cdot y}{\pi}
\end{aligned}\]

\hypertarget{p1}{%
\subsubsection{p1}\label{p1}}

Therefore, we can rewrite \(p_{1}\) using the Taylor method, as:

\[\begin{aligned}
p_{1} = \frac{\phi_{p} \cdot y}{2} + \frac{\phi_{p}}{2} y \cdot \hat{y_{t}} + \frac{\phi_{p} \cdot y}{\pi} \pi \cdot \hat{\pi_{t}} - \frac{\phi_{p} \cdot y}{\pi}\pi \cdot \hat{\pi_{t-1}}\\
= \frac{\phi_{p} \cdot y}{2} + \frac{\phi_{p} \cdot y \cdot \hat{y_{t}} }{2} + \phi_{p} \cdot y \cdot \hat{\pi}_{t} - \alpha \cdot \phi_{p} \cdot y \cdot \hat{\pi}_{t-1} \\
\end{aligned}\]

\hypertarget{p2}{%
\subsubsection{p2}\label{p2}}

and for \(p_{2}\)

\[\begin{aligned}
\left[ p_{2} \right]^{SS} &= \phi_{p} y
\\
\left[ \frac{dp_{2}}{dy_{t}} \right]^{SS} &= \phi_{p}
\\
\left[ \frac{dp_{2}}{d\pi_{t}} \right]^{SS} &= \frac{\phi_{p} y}{\pi}
\\
\left[ \frac{dp_{2}}{d\pi_{t-1}} \right]^{SS} &= - \alpha \frac{\phi_{p} y}{\pi}
\end{aligned}\]

Then, writing the Taylor approximation for \(p_{2}\)

\[\begin{aligned}
p_{2} &= \phi_{p} y + \phi_{p} y \hat{y}_{t} + \frac{\phi_{p} y}{\pi} \pi \hat{\pi}_{t} - \alpha \frac{\phi_{p} y}{\pi} \pi \hat{\pi}_{t-1}\\
&= \phi_{p} y + \phi_{p} y \hat{y}_{t} + \phi_{p} y \hat{\pi}_{t} - \alpha \phi_{p} y \hat{\pi}_{t-1}
\end{aligned}\]

\hypertarget{p3}{%
\subsubsection{p3}\label{p3}}

Lastly, repeating the same process for the last term \(p_{3}\)

\[\begin{aligned}
\left[ p_{3} \right]^{SS} &= \frac{\phi_{p} y} {2}\\\left[ \frac{dp_{3}}{dy_{t}} \right]^{SS} &= \frac{\phi_{p}}{2}
\end{aligned}\]

Then, writing the Taylor approximation for \(p_{3}\)

\[p_{3} = \frac{\phi_{p} y} {2} + \frac{\phi_{p}}{2}y \hat{y}_{t}\]

Putting it all together with \(y_{t} = c + p_{1} - p_{2} + p_{3}\) we
get:

\[\begin{aligned}
y_{t} = & \ c_t  +\left[ \overbrace{\frac{\phi_{p} \cdot y}{2}}^{A} + \overbrace{\frac{\phi_{p} \cdot y \cdot \hat{y_{t}} }{2}}^{B} + \overbrace{\phi_{p} \cdot y \cdot \hat{\pi}_{t}}^{C} - \overbrace{\alpha \cdot \phi_{p} \cdot y \cdot \hat{\pi}_{t-1}}^{D} \right] \\ & \ - \left[ \overbrace{\phi_{p} y}^{2A} + \overbrace{\phi_{p} y \hat{y}_{t}}^{2B} + \overbrace{ \phi_{p} y \hat{\pi}_{t}}^{C} - \overbrace{\alpha \phi_{p} y \hat{\pi}_{t-1}}^{D} \right] + \left[ \overbrace{\frac{\phi_{p} y} {2}}^{A} + \overbrace{\frac{\phi_{p}}{2}y \hat{y}_{t}}^{B} \right] \\= &\  c_{t} + A + B + C -D - 2A - 2B - C + D + A + B\\
\end{aligned}\]

Now, using
\(y = c + \frac{\phi_{p} \cdot y}{2} - \phi_{p} y + \frac{\phi_{p} y} {2} = c\),
we subtract y on both sides, such that

\[\begin{aligned}
A + B + C &- D - 2A - 2B - C + D + A + B = 0
\text{ and thus, }
\\
y_{t} - y &= c_t - c
\\
\Rightarrow y \cdot \hat{y}_t &= c \cdot \hat{c}_t
\\
\Rightarrow \hat{y}_t &= \hat{c}_t
\end{aligned}\]

\hypertarget{equation-19-in-paper}{%
\subsection{Equation (19) in paper:}\label{equation-19-in-paper}}

From equation (5)
\[\Lambda_t = \frac{a_t}{C_t-\gamma C_{t-1}} - \beta \gamma E_t \left[ \frac{a_{t+1}}{C_{t+1} - \gamma C_t} \right]\]

Using
\(c_{t}=C_{t} / Z_{t} \  \lambda_{t}=Z_{t} \cdot \Lambda_{t}, \text{ and } z_t = Z_t / Z_{t-1}\)
we can rewrite the above equation using its stationary variables:

\[\begin{aligned}
\lambda_{t} / Z_t =\frac{a_{t}}{Z_{t} c_{t}-\gamma Z_{t-1} c_{t-1}}-\beta \gamma E_{t} \left(\frac{a_{t+1}}{Z_{t+1} c_{t+1}-\gamma Z_{t-1} c_{t}} \right)\\
\text{ times each term with: } \left(Z_{t-1} / Z_{t-1}\right) \text{ and } (Z_{t} / Z_{t}) \text{ respectively} \\
\lambda_{t} =\frac{a_{t} z_{t}}{z_{t} c_{t}-\gamma c_{t-1}}-\beta \gamma E_{t}\left(\frac{a_{t+1}}{z_{t+1} c_{t+1}-\gamma c_{t}}\right)
\end{aligned}\]

Once again, for simplicity, we separate the equation into two parts:

\[\lambda_{t}=\frac{a_{t} z_{t}}{z_{t} c_{t}-\gamma c_{t-1}}-\beta \gamma E{t}\left(\frac{a_{t+1}}{z_{t+1} c_{t+1}-\gamma c_{t}}\right)\
= a_1 + \beta \gamma E_t (a_2)\]

Now we can determine the Taylor approximations individually:

\hypertarget{a1}{%
\subsubsection{a1}\label{a1}}

\[\begin{aligned}
\left[ a_{1} \right]^{SS} &=  \frac{a z}{z c -\gamma c} = \frac{a z}{(z  -\gamma)c }
\\
\left[\frac{da_{1}}{da_{t}} \right]^{SS} &= 
\frac{z}{(z  -\gamma)c}
\\
\left[\frac{da_{1}}{dz_{t}} \right]^{SS} &= 
\frac{a z c- \gamma a c-a z c}{(z c- \gamma c)^{2}}
\\
&= -\frac{ \gamma a c}{(z c- \gamma c)^{2}}
\\
\left[\frac{da_{1}}{dc_{t}} \right]^{SS} &= 
\frac{da_{1}}{dc_{t}} 
\left[ \frac{a_{t} z_{t}}{(z_t c_{t}-\gamma c_{t-1})} \right]^{SS}\\
&= -a_{t} z_t\left(z_{t} c_{t} - \gamma c_{t-1}\right)^{-2} \cdot z t\\
&= - \frac {a z z} {(z c - \gamma c)^{2}}\\
\left[\frac{da_{1}}{dc_{t-1}} \right]^{SS} &=
\left[ \frac{d}{d c_{t-1}} \left[ -a_{t} z_{t} 
\left( z_{t} c_{t} - \gamma c_{t-1} \right)^{-1} \right] \right]^{SS}\\
&= \left[ -a_{t} z_{t}\left(z_{t} c_{t}-\gamma c_{t-1}\right)^{-2}
\cdot(-\gamma) \right]^{SS}\\
&= \frac{a z \gamma}{(z c-\gamma c)^{2}}
\end{aligned}\]

The similarity between \(a_1\) and \(a_2\) means that some of the steps
will yield very similar results:

\hypertarget{a2}{%
\subsubsection{a2}\label{a2}}

\[\begin{aligned}
\left[ a_{2} \right]^{SS} &= \left[ a_{1} \right]^{SS} / z\\
&= \frac{a}{(z  -\gamma)c }\\
\left[\frac{da_{2}}{da_{t+1}} \right]^{SS} 
&= \left[\frac{da_{1}}{da_{t}} \right]^{SS} \cdot 1 / z\\
&= \frac{1}{(z  -\gamma)c}\\
\left[\frac{da_{2}}{dz_{t+1}} \right]^{SS} &= \left\{\frac{d}{d z_{t+1}}\left[\frac{a_{t+1}}{z_{t+1} c_{t+1}-\gamma c_{t}}\right]\right\}^{SS}\\
&=\left\{-a_{t+1}\left[z_{t+1} c_{t+1}-\gamma c_t\right]^{-2} \cdot c_{t+1}\right\}^{SS}\\
&= - \frac{a c}{z c-\gamma c}\\
\left[\frac{da_{2}}{dc_{t+1}} \right]^{SS} &= \left[\frac{da_{1}}{dc_{t}} \right]^{SS} \cdot 1 / z \\
&= - \frac {a z} {(z c - \gamma c)^{2}}\\\left[\frac{da_{2}}{dc_{t}} \right]^{SS} &= \left[\frac{da_{1}}{dc_{t-1}} \right]^{SS} \cdot 1 / z\\&= \frac{a \gamma}{(z c-\gamma c)^{2}}
\end{aligned}\]

Thus, we have
\[\lambda_t = a_1 - \beta \gamma E_t (a_2) \text{ which we can expand, and } 
\\
\lambda = \frac{a z}{(z  -\gamma)c } - \beta \gamma \frac{a}{(z-\gamma) c}\]

\hypertarget{lambda}{%
\subsubsection{Lambda}\label{lambda}}

\[\begin{aligned}
\lambda_t &= \overbrace{
\left[ 
\mathbf{\frac{a z}{(z  -\gamma)c }}
+ \frac{z}{(z  -\gamma)c} \cdot a \hat{a}_t 
- \frac{ \gamma a c}{(z c- \gamma c)^{2}} \cdot z \hat{z}_t 
- \frac{a z z}{(z c-\gamma c)^{2}} \cdot c \hat{c}_t
+ \frac{a z \gamma}{(z c-\gamma c)^{2}} \cdot c \hat{c}_{t-1} 
\right]
}^{a_1}\\
&  - \beta \gamma E_t
\overbrace{
\left[
\mathbf{\frac{a}{(z-\gamma) c}}
+ \frac{1}{(z-\gamma) c} \cdot a \hat{a}_{t+1}
- \frac{a c}{z c-\gamma c} z \hat{z}_{t+1}
- \frac{a z}{(z c-\gamma c)^{2}} \cdot c \hat{c}_{t+1}
+ \frac{a \gamma}{(z c-\gamma c)^{2}} \cdot c \hat{c}_t
\right]}^{a_2}
\end{aligned}\]

subtracting \(\lambda\) from both sides of the equations to get
\(\lambda_t - \lambda\) on the left side, we get:

\[\begin{aligned}
\lambda \hat{\lambda}_t &=
\left[ 
\frac{z}{(z  -\gamma)c} \cdot a \hat{a}_t 
- \frac{ \gamma a c}{(z c- \gamma c)^{2}} \cdot z \hat{z}_t 
- \frac{a z z}{(z c-\gamma c)^{2}} \cdot c \hat{c}_t
+ \frac{a z \gamma}{(z c-\gamma c)^{2}} \cdot c \hat{c}_{t-1} 
\right]\\
&  - \beta \gamma E_t
\left[
\frac{1}{(z-\gamma) c} \cdot a \hat{a}_{t+1}
- \frac{a c}{z c-\gamma c} z \hat{z}_{t+1}
- \frac{a z}{(z c-\gamma c)^{2}} \cdot c \hat{c}_{t+1}
+ \frac{a \gamma}{(z c-\gamma c)^{2}} \cdot c \hat{c}_t
\right]\\ 
&= \frac{az}{(z  -\gamma)c} \cdot  \hat{a}_t
- \frac{ \gamma a c z}{(z c- \gamma c)^{2}} \cdot \hat{z}_t 
- \frac{a c z z}{(z c-\gamma c)^{2}} \cdot \hat{c}_t
+ \frac{\gamma a c z }{(z c-\gamma c)^{2}} \cdot \hat{c}_{t-1} \\
& - \frac{\beta \gamma a}{(z-\gamma) c} \cdot E_t \hat{a}_{t+1}
+ \frac{a c z}{z c-\gamma c} E_t \hat{z}_{t+1}
+ \frac{a c z}{(z c-\gamma c)^{2}} \cdot E_t\hat{c}_{t+1}
- \frac{ \gamma a c }{(z c-\gamma c)^{2}} \cdot \hat{c}_t
\end{aligned}\]

From the paper we know that \(a=1\),
\(E_t \hat{a}_{t+1} = (\rho_a \hat{a}_t)\), and \(E_t \hat{z}_{t+1}=0\),
and thus:

\[\begin{aligned}
\lambda \hat{\lambda}_t &= 
\frac{z}{(z  -\gamma)c} \cdot  \hat{a}_t
- \frac{ \gamma c z}{(z c- \gamma c)^{2}} \cdot \hat{z}_t 
- \frac{c z z}{(z c-\gamma c)^{2}} \cdot \hat{c}_t
+ \frac{\gamma c z }{(z c-\gamma c)^{2}} \cdot \hat{c}_{t-1} 
\\
& - \frac{\beta \gamma a}{(z-\gamma) c} \cdot \rho_a \hat{a}_t
+ 0
+ \beta \gamma \frac{c z}{(z c-\gamma c)^{2}} \cdot E_t\hat{c}_{t+1}
- \beta \gamma \frac{\gamma c }{(z c-\gamma c)^{2}} \cdot \hat{c}_t
\end{aligned}\]

The common denominator can be removed by multiplying both sides by
\((zc-\gamma c)^2\) and then we divide by c

\[\begin{aligned}
\lambda \hat{\lambda}_t (zc-\gamma c)^2 &= 
z (zc-\gamma c) \hat{a}_t
- \gamma c z \hat{z}_t 
- c z z \hat{c}_t
+ \gamma c z \hat{c}_{t-1} 
\\
& - \beta \gamma a  \rho_a (zc-\gamma c) \hat{a}_t
+ c zE_t\hat{c}_{t+1}
- \gamma c \hat{c}_t\\
\lambda \hat{\lambda}_t (z-\gamma )^2c &=
z (z-\gamma ) \hat{a}_t
- \gamma z \hat{z}_t 
-  z z \hat{c}_t
+ \gamma  z \hat{c}_{t-1} \\
& - \beta \gamma a  \rho_a (z-\gamma ) \hat{a}_t
+ \beta \gamma zE_t\hat{c}_{t+1}
- \gamma \hat{c}_t
\end{aligned}\]

We know \(\hat{c}_t = \hat{y}_t\) and \(c = y\)

\[\begin{aligned}
\lambda \hat{\lambda}_t (z-\gamma)^{2} y &=
(z-\gamma) (z-\beta \gamma \rho_a ) \hat{a}_t
-\gamma z \hat{z}_t
-\left(z^{2} + \beta \gamma^{2}\right) \hat{y}_{t}
+ \gamma z \hat{y}_{t-1}
+ \beta \gamma z E_{t} \hat{y}_{t+1}
\end{aligned}\]

From the Appendix, we can use the fact that

\[y \lambda=\frac{(z-\beta \gamma)}{(z-\gamma)}\]

\[\begin{aligned} &(z-\gamma)(z-\beta \gamma) \hat{\lambda}{t}\\ 
& = (z-\gamma)\left(z-\beta \gamma \rho_{a}\right) \hat{a}_{t}-\gamma z \hat{z}_{t}-\left(z^{2}+\beta \gamma^{2}\right) \hat{y}_{t}+z \gamma \hat{y}_{t-1} + \beta \gamma z E_{t} \hat{y}_{t+1}
\end{aligned}\]

yielding the final result:
\[(z-\gamma)(z-\beta \gamma) \hat{\lambda}_{t}=(z-\gamma)\left(z-\beta \gamma p_{a}\right) \hat{a}_{t}-\gamma z \hat{z}_{t}+z \gamma \hat{y}_{t-1}+\beta \gamma z E_{t} \hat{y}_{t+1}-\left(z^{2}+\beta \gamma^{2}\right) \hat{y}_{t}\]

\hypertarget{equation-20}{%
\subsection{Equation (20)}\label{equation-20}}

From Equation (7)

\[\lambda_{t}=\beta r_{t} E_{t}\left(\frac{\lambda_{t+1}}{z_{t+1} \pi_{t+1}}\right)\]

The steady state equation follows as

\[\lambda = \beta r \frac{\lambda} {z \pi} \\
\Rightarrow 1=\frac{\beta r} {z \pi}\]

Now, substituting each variable in the fashion similar to
\(\lambda_t = \lambda e^{\hat{\lambda}_t}\)

\[\begin{array}{l}
\lambda_{t}=\beta r_{t} E_{t}\left[\frac{\lambda_{t+1}}{z_{t+1} \pi_{t+1}}\right]\\
\Rightarrow \lambda e^{\hat{\lambda}_t}=\beta r e^{\hat{r}_{t}} E_{t}\left[\frac{\lambda e^{\hat{\lambda}_{t+1}}}{z e^{\hat{z}_{t+1}} \pi e^{\hat{\pi}_{t+1}}} \right]\\
\Rightarrow \lambda\left(1+\hat{\lambda}{t}\right)=\frac{\beta r \lambda}{z \pi} \quad E_{t}\left[\frac{\left(1+\hat{\lambda}_{t+1}-\hat{z}_{t+}-\hat{\pi}_{t+1}+\hat{r}_{t}\right)}{1}\right]\\
\Rightarrow \left(1+\hat{\lambda}_{t}\right)= E_{t}\left[1+\hat{\lambda}_{t+1}-\hat{z}_{t+}-\hat{\pi}_{t+1}+\hat{r}_{t}\right]\\
\text{using }{ E_t }[\hat{z}_{t+1}]=0\\
\Rightarrow 1+\hat{\lambda}_{t} = 1+E_{t}\left[\hat{\lambda}_{t+1}-\hat{\pi}_{t+1}+\hat{r}_{t}\right]\\
\therefore \hat{\lambda}_{t}=\hat{r}_{t}+E_{t}\left[\hat{\lambda}_{t+1} - \hat{\pi}_{t+1}\right]
\end{array}\]

\hypertarget{equation-21}{%
\subsection{Equation (21)}\label{equation-21}}

\[1=\frac{z_{t}}{z_{t} q_{t}-\gamma q_{t-1}}-\beta \gamma E_{t}\left[\left(\frac{a_{t+1}}{a_{t}}\right)\left(\frac{1}{z_{t+1} q_{t+1}-\gamma q_{t}}\right)\right]\]

The LHS of the equation, instead of approximating, the log of 1 is zero,
so that is used instead For the RHS, we continue as before, by
separating the function into simpler terms

\[\begin{aligned}
z_1 = \frac{z_{t}}{z_{t} q_{t}-\gamma q_{t-1}} \\
z_2 = \left(\frac{a_{t+1}}{a_{t}}\right)\left(\frac{1}{z_{t+1} q_{t+1}-\gamma q_{t}}\right)
\end{aligned}\]

\hypertarget{z1}{%
\subsubsection{Z1}\label{z1}}

\[\begin{aligned}
z_{1}^{SS}&= \frac{z}{z q-\gamma q} \\
\left[\frac{dz_{1}}{dz_{t}} \right]^{SS} &=
\left[\frac{\left(z_{t} q_{t}-\gamma q_{t-1}\right)-z_{t}\left(q_{t}\right)}{\left(z_{t} q_{t}-\gamma 
q_{t-1}\right)^{2}}\right]^{SS}\\
&=\frac{(z q-\gamma q)-z q}{(z q-\gamma q)}\\
&=-\frac{\gamma q}{(zq-\gamma q)^2}\\
\left[\frac{dz_{1}}{dq_{t}} \right]^{SS} &= \left[-z_{t}\left(z_{t} q_{t}-\gamma q_{t-1}\right)^{-2} z_{t} \right]^{SS} \\
&= -\frac{z^{2}}{\left(z q-\gamma q\right)^2}\\
\left[\frac{dz_{1}}{dq_{t-1}} \right]^{SS} &= \left[-z_{t}\left(z_{t} q_{t}-\gamma q_{t-1}\right)^{-2}(-\gamma)\right]^{SS} \\
&=\frac{z \gamma}{(z q-\gamma q)^2}
\end{aligned}\]

Now we can expand the linearisation equation for \(z_1\)

\[z_1 = \frac{z}{z q-\gamma q} 
- \frac{\gamma q}{(z q-\gamma q)^2} z \cdot \hat{z}_t
- \frac{z^{2}}{(z q-\gamma q)^2} q \cdot \hat{q}_t
+ \frac{z \gamma}{(z q-\gamma q)^2} q \cdot \hat{q}_{t-1}\]

\hypertarget{z2}{%
\subsubsection{Z2}\label{z2}}

\[\begin{aligned}
z_{2}^{SS} &= \left(\frac{a}{a}\right)\left(\frac{1}{z q-\gamma q}\right) \\
&= \frac{1}{z q-\gamma q}\\
\left[\frac{dz_{1}}{da_{t+1}} \right]^{SS} &= \left(\frac{1}{a}\right)\left(\frac{1}{z q-\gamma q}\right)\\
\left[\frac{dz_{1}}{da_{t}} \right]^{SS} &= 
- \left(\frac{a}{a^{2}}\right)\left(\frac{1}{z q-\gamma q}\right)\\
&= -\left(\frac{1}{a}\right)\left(\frac{1}{z q-\gamma q}\right)\\
\left[\frac{dz_{1}}{dz_{t+1}} \right]^{SS} &= \left[-\left(\frac{a_{t+1}}{a_{t}}\right)\left(z_{t+1} q_{t+1}-\gamma q_{t}\right)^{-2} \cdot\left(q_{t+1}\right) \right]^{SS} \\
&=-\left(\frac{a}{a}\right) \frac{q}{(zq-\gamma q)^2}\\
&= -\frac{q}{(zq-\gamma q)^2}\\
\left[\frac{dz_{1}}{dq_{t+1}} \right]^{SS} &= -\left[\left(\frac{a_{t+1}}{a_{t}}\right)\left(z_{t+1} q_{t+1}-\gamma q_{t}\right)^{-2}\left(z_{t+1}\right)\right]^{SS} \\
&=-\left(\frac{a}{a}\right) \frac{z}{(z q-\gamma q)^{2}} \\
&=-\frac{z}{(z q-\gamma q)^{2}} \\
\left[\frac{dz_{1}}{dq_{t}} \right]^{SS} &= \left[\left(\frac{a_{t+1}}{a_{t}}\right)\left(z_{t+1} q_{t+1}-\gamma q_{t}\right)^{-2}(-\gamma)\right]^{SS} \\
&=\frac{\gamma}{(z q-\gamma q)^{2}}
\end{aligned}\]

Therefore, we have \(z_2\)

\{Recall a=1\}

\[\begin{aligned}
z_2 =& \frac{1}{z q-\gamma q}
+ \left(\frac{1}{a}\right)\left(\frac{1}{z q-\gamma q}\right) \cdot a \hat{a}_{t+1}
-\left(\frac{1}{a}\right)\left(\frac{1}{z q-\gamma q}\right) \cdot a \hat{a}_{t}\\
&-\left(\frac{1}{a}\right)\left(\frac{1}{z q-\gamma q}\right) \cdot z\hat{z}_{t}
-\frac{q}{(zq-\gamma q)^2} \cdot q \hat{q}_{t+1}
+\frac{\gamma}{(z q-\gamma q)^{2}} \cdot q \hat{q}_{t}\\
=& \frac{1}{z q-\gamma q}
+ \left(\frac{1}{z q-\gamma q}\right) \cdot \hat{a}_{t+1}
-\left(\frac{1}{z q-\gamma q}\right) \cdot \hat{a}_{t}\\
&-\left(\frac{1}{z q-\gamma q}\right) \cdot z\hat{z}_{t+1}
-\frac{q^2}{(zq-\gamma q)^2} \cdot \hat{q}_{t+1}
+\frac{\gamma}{(z q-\gamma q)^{2}} \cdot q \hat{q}_{t}
\end{aligned}\]

\newpage

Now for \(\beta \gamma E_t z_2\), using
\(E_t \hat{a}_{t+1}=\rho_a\hat{a}_t\)

\[\begin{aligned}
\beta \gamma E_t z_2
=& \beta \gamma \left[ \frac{1}{z q-\gamma q}
+ \left(\frac{1}{z q-\gamma q}\right) \cdot E_t\hat{a}_{t+1}
-\left(\frac{1}{z q-\gamma q}\right) \cdot E_t\hat{a}_{t}
-\left(\frac{1}{z q-\gamma q}\right) \cdot zE_t\hat{z}_{t+1} \right.\\
& \left. -\frac{q^2}{(zq-\gamma q)^2} \cdot E_t\hat{q}_{t+1}
+\frac{\gamma}{(z q-\gamma q)^{2}} \cdot q E_t\hat{q}_{t} \right]\\
=& \beta \gamma \left[ \frac{1}{z q-\gamma q}
+ \left(\frac{1}{z q-\gamma q}\right) \cdot \rho_a\hat{a}_t
-\left(\frac{1}{z q-\gamma q}\right) \cdot \hat{a}_{t}
-0
-\frac{q^2}{(zq-\gamma q)^2} \cdot E_t\hat{q}_{t+1}
+\frac{\gamma}{(z q-\gamma q)^{2}} \cdot q \hat{q}_{t} \right]\\
=& \frac{\beta \gamma}{z q-\gamma q}
+ \left(\frac{\beta \gamma}{z q-\gamma q}\right) \cdot \rho_a\hat{a}_t
-\left(\frac{\beta \gamma}{z q-\gamma q}\right) \cdot \hat{a}_{t}
-\frac{\beta \gamma q^2}{(z q-\gamma q)^2} \cdot E_t\hat{q}_{t+1}
+\frac{\beta \gamma^2}{(z q-\gamma q)^{2}} \cdot q \hat{q}_{t}
\end{aligned}\]

\hypertarget{together}{%
\subsubsection{Together}\label{together}}

Now, we can write out \(0 = z_1 - \beta \gamma E_t z_2\)

Using

\[z_1-z_1^{SS} - \beta \gamma (z_2 - z_2^{SS}) \approx z_1 \hat{z_1} - z_2 \hat{z_2}\\
\Rightarrow \text{subtract from the RHS: } \frac{z}{z q-\gamma q}-\beta \gamma \frac{a}{a z q-\gamma a q}\]

\[\begin{aligned}
0 = &\frac{z}{z q-\gamma q} 
- \frac{\gamma q}{(z q-\gamma q)^2} z \cdot \hat{z}_t
- \frac{z^{2}}{(z q-\gamma q)^2} q \cdot \hat{q}_t
+ \frac{z \gamma}{(z q-\gamma q)^2} q \cdot \hat{q}_{t-1}\\
&-\left[
\frac{\beta \gamma}{z q-\gamma q}
+ \left(\frac{\beta \gamma}{z q-\gamma q}\right) \cdot \rho_a\hat{a}_t
-\left(\frac{\beta \gamma}{z q-\gamma q}\right) \cdot \hat{a}_{t}
-\frac{\beta \gamma q^2}{(z q-\gamma q)^2} \cdot E_t\hat{q}_{t+1}
+\frac{\beta \gamma^2}{(z q-\gamma q)^{2}} \cdot q \hat{q}_{t} \right]
\end{aligned}\]

Multiply both sides by \((z q-\gamma q)^{2} \frac{1}{q}\)

\[\begin{aligned}
0 = & z \cdot (z -\gamma )
- \gamma z \cdot \hat{z}_t 
- z^{2} \cdot \hat{q}_t 
+ z \gamma \cdot \hat{q}_{t-1}\\
&-\left[
\beta \gamma(z -\gamma )
+ \beta \gamma \cdot \rho_a\hat{a}_t (z -\gamma )
- \beta \gamma (z -\gamma ) \cdot \hat{a}_{t}
- \beta \gamma q \cdot E_t\hat{q}_{t+1}
+ \beta \gamma^2 \cdot \hat{q}_{t} \right]\\
0 = & z^{2}-\gamma z
- \gamma z \cdot \hat{z}_t 
- z^{2} \cdot \hat{q}_t 
+ z \gamma \cdot \hat{q}_{t-1}\\
&-\beta \gamma z+\beta \gamma^{2}-\beta \gamma \rho_{a} z \hat{a}_{t}+\beta \gamma^{2} \rho_{a} \hat{a}_{t}\\
&+\beta \gamma z \hat{a}_{t}-\beta \gamma^{2} \hat{a}_{t}
+\beta \gamma q E_{t} \hat{q}_{t+1}-\beta \gamma^{2} \hat{q}_{t}
\end{aligned}\]

Finally, we subtract:

\[\left[\frac{z}{z q-\gamma q}-\beta \gamma \frac{a}{a z q-\gamma a q} \right] \cdot (z q-\gamma q)^{2} \frac{1}{q}\\
= z(z -\gamma) -\beta \gamma (z- \gamma)\\
=z^2 -\gamma z-\beta \gamma z + \beta \gamma^2\]

to yield the final solution:

\[\begin{aligned}
0 = & \begingroup\color{red}z^{2} \endgroup - \begingroup \color{red} {\gamma z} \endgroup - \gamma z \cdot \hat{z}_t - z^{2} \cdot \hat{q}_t 
+ z \gamma \cdot \hat{q}_{t-1}\\
& - \begingroup \color{red}{\beta \gamma z} \endgroup + \begingroup \color{red}{\beta \gamma^{2}} \endgroup - \beta \gamma \rho_{a} z \hat{a}_{t} + \beta \gamma^{2} \rho_{a} \hat{a}_{t}\\
& + \beta \gamma z \hat{a}_{t} - \beta \gamma^{2} \hat{a}_{t}
+\beta \gamma q E_{t} \hat{q}_{t+1}-\beta \gamma^{2} \hat{q}_{t} \\
& - \begingroup \color{red}{z^2} \endgroup + \begingroup \color{red}{\gamma z} \endgroup + \begingroup \color{red}{\beta \gamma z} \endgroup - \begingroup \color{red}{\beta \gamma^2} \endgroup \\
=& - \gamma z \cdot \hat{z}_t 
- z^{2} \cdot \hat{q}_t 
+ z \gamma \cdot \hat{q}_{t-1}\\
&-\beta \gamma \rho_{a} z \hat{a}_{t}+\beta \gamma^{2} \rho_{a} \hat{a}_{t}\\
&+\beta \gamma z \hat{a}_{t}-\beta \gamma^{2} \hat{a}_{t}
+\beta \gamma q E_{t} \hat{q}_{t+1}-\beta \gamma^{2} \hat{q}_{t} \\
= & \hat{z}_{t}(-\gamma z)+\hat{q}_{t}\left(-z^{2}-\beta \gamma^{2}\right)+\hat{q}_{t-1}(z \gamma) \\
&+\hat{a}_{t}\left(-\beta \gamma \rho_{a} z+\beta \gamma^{2} p_{a}+\beta \gamma z-\beta \gamma^{2}\right) \\
&+\beta \gamma q E_{t} \hat{q}_{t+1}\\
= & \gamma z \hat{q}_{t-1}-\hat{q}_{t}\left(z^{2}+\beta \gamma^{2}\right)+\beta \gamma q E_{t} \hat{q}_{t+1}+\hat{a}_{t} \beta \gamma\left(1-\rho_{a}\right)(z-\gamma)-\gamma z \hat{z}_{t}
\end{aligned}\]

\hypertarget{equation-22}{%
\subsection{Equation (22)}\label{equation-22}}

\[\begin{aligned} x_{t}=y_{t} / q_{t} \\
\text{Steady State: } x = y/p \end{aligned}\]

Now,

\[\begin{aligned} x_{t}=y_{t} / q_t\\
\Rightarrow x e^{\hat{x} t}=\frac{y e^{\hat{y} t}}{q e^{\hat{q} t}}\\
\Rightarrow\left(1+\hat{x}_{t}\right)=\left(1+\hat{y}_{t}-\hat{q}_{t}\right)\\
\therefore \quad \hat{x}_{t}=\hat{y}_{t}-\hat{q}_{t} \end{aligned}\]

\hypertarget{equation-23}{%
\subsection{Equation (23)}\label{equation-23}}

Starting from equation (13)

\[\begin{array}{l} (1-\theta_t)(\frac{P_t(i)}{P_t})^{-\theta_t} + \theta_t(\frac{P_t(i)}{P_t})^{-\theta_t-1}(\frac{W_t}{P_t})(\frac{1}{Z_t}) - \phi_p[(\frac{P_t(i)}{\pi_{t-1}^a \pi^{1-a}P_{t-1}(i)}-1)(\frac{P_t}{\pi_{t-1}^a \pi^{1-a}P_{t-1}(i)})] \\
+\beta\phi_pE_t(\frac{\Lambda_{t+1}}{\Lambda_{t}})(\frac{P_{t+1}(i)}{\pi_t^a\pi^{1-a}P_{t}(i)}-1)(\frac{P_{t+1}(i)}{\pi_t^a\pi^{1-a}P_{t}(i)})(\frac{Y_{t+1}}{Y_t})(\frac{P_t}{P_t(i)})=0 \end{array}\]

Substituting for the equilibrium Conditions and using (6):

\[\begin{aligned} (1-\theta_t) + \theta_t (\frac{W_t}{P_t})(\frac{1}{Z_t}) - \phi_p[(\frac{P_t}{\pi_{t-1}^a \pi^{1-a}P_{t-1}}-1)(\frac{P_t}{\pi_{t-1}^a \pi^{1-a}P_{t-1}})] \\
+\beta\phi_pE_t(\frac{\Lambda_{t+1}}{\Lambda_{t}})(\frac{P_{t+1}}{\pi_t^a\pi^{1-a}P_{t}}-1)(\frac{P_{t+1}}{\pi_t^a\pi^{1-a}P_{t}})(\frac{Y_{t+1}}{Y_t})=0 \end{aligned}\]

\[\begin{aligned} \Rightarrow \theta_t - 1 = & \theta_t \frac{a_t}{\Lambda_t Z_t} - \phi_p[(\frac{\pi_t}{\pi_{t-1}^a \pi^{1-a}}-1)(\frac{\pi_t}{\pi_{t-1}^a \pi^{1-a}})] \\
& +\beta\phi_pE_t(\frac{\Lambda_{t+1}Y_{t+1}}{\Lambda_{t}Y_t})(\frac{\pi_{t+1}}{\pi_t^a\pi^{1-a}}-1)(\frac{\pi_{t+1}}{\pi_t^a\pi^{1-a}}) \end{aligned}\]

\[\begin{aligned} \therefore \theta_{t}-1=& \theta_{t}\left(\frac{a_{t}}{\lambda_{t}}\right)-\phi_{p}\left(\frac{\pi_{t}}{\pi_{t-1}^{\alpha} \pi^{1-\alpha}}-1\right)\left(\frac{\pi_{t}}{\pi_{t-1}^{\alpha} \pi^{1-\alpha}}\right) \\
&+\beta \phi_{p} E_{t}\left[\left(\frac{\lambda_{t+1} y_{t+1}}{\lambda_{t} y_{t}}\right)\left(\frac{\pi_{t+1}}{\pi_{t}^{\alpha} \pi^{1-\alpha}}-1\right)\left(\frac{\pi_{t+1}}{\pi_{t}^{\alpha} \pi^{1-\alpha}}\right)\right]
\end{aligned}\]

In Steady State this becomes

\[\begin{aligned}
\quad \ \theta-1=& \theta\left(\frac{a}{\lambda}\right)-\phi_{p}\left(\frac{\pi}{\pi^{\alpha} \pi^{1-\alpha}}-1\right)\left(\frac{\pi}{\pi^{\alpha} \pi^{1-\alpha}}\right) \\
&+\beta \phi_{p} E\left[\left(\frac{\lambda y}{\lambda y}\right)\left(\frac{\pi}{\pi^{\alpha} \pi^{1-\alpha}}-1\right)\left(\frac{\pi}{\pi^{\alpha} \pi^{1-\alpha}}\right)\right]\\
\Rightarrow \theta-1=& \theta\left(\frac{1}{\lambda}\right)-0 + 0 \\
\therefore \quad \lambda = \frac{\theta}{\theta-1}
\end{aligned}\]

Using the above result we can obtain and simplify further. Then,
log-linearise using Uhlig's Method:

First we separate the four terms, for simplicity:

\hypertarget{term-1}{%
\subsubsection{Term 1:}\label{term-1}}

\[\begin{aligned} \theta_{t}-1 & = \theta e^{\hat{\theta}_{t}} \\
& = \theta\left(1+\hat{\theta}_{t}\right) \end{aligned}\]

\hypertarget{term-2}{%
\subsubsection{Term 2:}\label{term-2}}

\[\begin{aligned} \theta_{t}\left(\frac{a_{t}}{\lambda_{t}}\right) &= \theta e^{\hat{\theta}_{t}}\left(\frac{a e^{\hat{a}_{t}}}{\lambda \hat{e}^{\hat{\lambda}_t}}\right)\\
&= \frac{\theta a}{\lambda}\left(1+\hat{\theta}_{t}+\hat{a}_{t}-\hat{\lambda}_{t}\right)\\
& = \frac{\theta}{\lambda}\left(1+\hat{\theta}_{t} \right)+ \frac{\theta}{\lambda} \left(\hat{a}_{t}-\hat{\lambda}_{t}\right)\end{aligned}\]

\hypertarget{term-3}{%
\subsubsection{Term 3:}\label{term-3}}

\[\begin{aligned}
-\phi_{p}\left(\frac{\pi_{t}}{\pi_{t-1}^{\alpha} \pi^{1-\alpha}}-1\right)\left(\frac{\pi_{t}}{\pi_{t-1}^{\alpha} \pi^{1-\alpha}}\right) &= -\phi_{p}\left(\frac{\pi e^{\pi_{t}}}{\pi^{\alpha} e^{\alpha \hat{\pi}_{t-1}} \pi^{1-\alpha}}-1\right)\left(\frac{\pi e^{\hat{\pi}_t}}{\pi^{\alpha} e^{\alpha \hat{\pi}_{t-1}} \pi^{1-\alpha}}\right)\\
&= -\phi_p\left[\exp \left[2 \hat{\pi}_{t}-2 \alpha \hat{\pi}_{t-1}\right]-\exp \left[\hat{\pi}_{t}-\hat{\pi}_{t-1}\right]\right]\\
&=-\phi_p\left[\hat{\pi}_{t}-\alpha \hat{\pi}_{t-1}\right]
 \end{aligned}\]

\hypertarget{term-4}{%
\subsubsection{Term 4:}\label{term-4}}

\[\begin{aligned}
&\beta \phi_{p} E_{t}\left[\left(\frac{\lambda_{t+1} y_{t+1}}{\lambda_{t} y_{t}}\right)\left(\frac{\pi_{t+1}}{\pi_{t}^{\alpha} \pi^{1-\alpha}}-1\right)\left(\frac{\pi_{t+1}}{\pi_{t}^{\alpha} \pi^{1-\alpha}}\right)\right] \\
=& \beta \phi_{p} E_{t}\left[\left(\frac{\lambda e^{\hat{\lambda}_{t+1}} \cdot y e^{\hat{y}_{t+1}}}{\lambda e^{\hat{\lambda}_t} y e^{\hat{y}_t}}\right)\right]\left(\frac{\pi e^{\pi_{t+1}}}{\pi^{\alpha} e^{\alpha \hat{\pi}_{t}} \pi^{1-\alpha}}-1\right)\left(\frac{\pi e^{\hat{\pi}_{t+1}}}{\pi^{\alpha} e^{\alpha \hat{\pi}_{t}} \pi^{1-\alpha}}\right)\\
=& \beta \phi_{p} E_{t}\left[\exp \left(\hat{\lambda}_{t+1}+\hat{y}_{t+1}-\hat{\lambda}_{t}-\hat{y}_{t}+2 \cdot \hat{\pi}_{t+1}-2 \alpha \hat{\pi}_{t}\right)\right.\\
&\left.-\exp \left(\hat{\lambda}_{t+1}+\hat{y}_{t+1}-\hat{\lambda}_{t}-\hat{y}_{t}+\hat{\pi}_{t+1}-\alpha \hat{\pi}_{t}\right)\right] \\
=& \beta \phi_{p} E_{t}\left[ \hat{\pi}_{t+1}-\alpha \hat{\pi}_{t}\right]
\end{aligned}\]

Putting all the components together then yields:

\[ \theta\left(1+\hat{\theta}_{t}\right)=\frac{\theta}{\lambda}\left(1+\hat{\theta}_{t} \right)+ \frac{\theta}{\lambda} \left(\hat{a}_{t}-\hat{\lambda}_{t}\right) - \phi_p\left[\hat{\pi}_{t}-\alpha \hat{\pi}_{t-1}\right] + \beta \phi_{p} E_{t}\left[ \hat{\pi}_{t+1}-\alpha \hat{\pi}_{t}\right] \]

Simplify using
\[\lambda=\frac{\theta}{\theta-1}, \quad \hat{e}_{t}=-\left(\frac{1}{\phi_{p}}\right) \hat{\theta}_{t} \text {, and } \psi=\frac{(\theta-1)}{\phi_{p}}\]
i.e.
\[ 1 - \frac{1}{\lambda} = 1/\theta, \\ \frac{\theta}{\lambda} = \theta - 1 \]

\[\begin{aligned} \theta \left(1+\hat{\theta}_{t}\right)\cdot \left[ 1 - \frac{1}{\lambda} \right] =& \frac{\theta}{\lambda} \left(\hat{a}_{t}-\hat{\lambda}_{t}\right) - \phi_p\left[\hat{\pi}_{t}-\alpha \hat{\pi}_{t-1}\right] + \beta \phi_{p} E_{t}\left[ \hat{\pi}_{t+1}-\alpha \hat{\pi}_{t}\right]\\
 \frac{\left(1+\hat{\theta}_{t}\right)}{\phi_{p}} =& \frac{\theta - 1}{\phi_{p}} \left(\hat{a}_{t}-\hat{\lambda}_{t}\right) - \left[\hat{\pi}_{t}-\alpha \hat{\pi}_{t-1}\right] + \beta E_{t}\left[ \hat{\pi}_{t+1}-\alpha \hat{\pi}_{t}\right]
\end{aligned}\]

Ignoring the constant parameter \(-1/\phi_p\), and rearranging the terms
yields the final result:

\[\begin{aligned} &-\hat{e}_t =  \psi \left(\hat{a}_{t}-\hat{\lambda}_{t}\right) -(1+\beta \alpha)\hat{\pi}_{t} +\alpha \hat{\pi}_{t-1} + \beta E_{t}\left[ \hat{\pi}_{t+1}\right]\\
\Rightarrow &(1+\beta \alpha) \hat{\pi}_{t}=\alpha \hat{\pi}_{t-1}+\beta E_{t} \hat{\pi}_{t+1}-\psi \hat{\lambda}_{t}+\psi \hat{a}_{t}+\hat{e}_{t}
\end{aligned}\]

\hypertarget{equation-24}{%
\subsection{Equation (24)}\label{equation-24}}

\[\ln \left(r_{t} / r\right)=\rho_{r} \ln \left(r_{t-1} / r\right)+\rho_{\pi} \ln \left(\pi_{t-1} / \pi\right)+\rho_{x} \ln \left(x_{t-1} / x\right)+\varepsilon_{r t}\]

Directly applying Uhlig's method using \(r_t=r\cdot e^{\hat{r}_t}\) for
each variable of the equation yields the required result:

\[\begin{aligned} & \ln \left[\frac{r e^{\hat{r} t}}{r}\right]=\operatorname{\rho_r} \ln \left[\frac{r e^{\hat{r}_{t-1}}}{r}\right]+\rho{\pi} \ln \left[\frac{\pi e^{\hat{\pi}_{t-1}}}{n}\right)+\rho_x \ln \left(\frac{x e^{\hat{x}_{t+1}}}{x}\right)+\varepsilon_{rt} \\
& \therefore \quad \hat{r}_{t}=\rho_{r} \hat{r}_{t-1}+\rho_{\pi} \hat{\pi}_{t-1}+\rho x \hat{x}_{t-1}+\varepsilon_{r t} \end{aligned}\]

\hypertarget{equation-25}{%
\subsection{Equation (25)}\label{equation-25}}

Starting Form Equation (9) and using the conditions from section
\protect\hyperlink{c}{C:}

\[ \begin{aligned}
&\frac{a_{t}}{\delta}\left[\ln \left(m^{*}\right)-\ln \left(m_{t}\right)+\ln \left(u_{t}\right)\right]-a_{t}\left(\frac{\phi_{m}}{2}\right)\left(\frac{Z_t m_t }{z Z_{t-1}m_{t-1}}-1\right)^{2} \\
&-a_{t} \phi_{m}\left(\frac{Z_t m_t }{z Z_{t-1}m_{t-1}} - 1\right)\left(\frac{Z_t m_t }{z Z_{t-1}m_{t-1}}\right) \\
&+\beta \phi_{m} E_{t}\left[a_{t+1}\left(\frac{Z_{t+1} m_{t+1} }{z Z_{t}m_{t}}-1\right)\left(\frac{Z_{t+1} m_{t+1} }{z Z_{t}m_{t}}\right)^{2}\left(\frac{z }{z_{t+1}}\right)\right] \\
&=Z_{t} \Lambda_{t}\left(1-\frac{1}{r_{t}}\right)\\
\Rightarrow \ &\frac{a_{t}}{\delta}\left[\ln \left(m^{*}\right)-\ln \left(m_{t}\right)+\ln \left(u_{t}\right)\right]-a_{t}\left(\frac{\phi_{m}}{2}\right)\left(\frac{z_{t} m_{t}}{z m_{t-1}}-1\right)^{2} \\
&-a_{t} \phi_{m}\left(\frac{z_{t} m_{t}}{z m_{t-1}}-1\right)\left(\frac{z_{t} m_{t}}{z m_{t-1}}\right) \\
&+\beta \phi_{m} E_{t}\left[a_{t+1}\left(\frac{z_{t+1} m_{t+1}}{z m_{t}}-1\right)\left(\frac{z_{t+1} m_{t+1}}{z m_{t}}\right)^{2}\left(\frac{z}{z_{t+1}}\right)\right] \\
&=\lambda_{t}\left(1-\frac{1}{r_{t}}\right) \\
\end{aligned}\]

\newpage

As indicated in the Appendix of the paper, substituting for the steady
state conditions in equation (9) yields

\[\begin{aligned}
&\frac{a}{\delta}\left[\ln \left(m^{*}\right)-\ln \left(m\right)+\ln \left(u\right)\right]-a\left(\frac{\phi_{m}}{2}\right)\left(\frac{z m}{z m}-1\right)^{2} \\
&-a \phi_{m}\left(\frac{z m}{z m_{t-1}}-1\right)\left(\frac{z m}{z m_{t-1}}\right) \\
& +\beta \phi_{m} E\left[a\left(\frac{z m}{z m}-1\right)\left(\frac{z m}{z m}\right)^{2}\left(\frac{z}{z}\right)\right] \\
&=\lambda\left(1-\frac{1}{r}\right)\\
\Rightarrow &\frac{1}{\delta}\left[\ln \left(m^{*}\right)-\ln \left(m\right)\right]-0-0+0=\lambda/r\left(r-1\right)\\
\therefore &\quad \ln (m) =\ln \left(m^{*}\right)-\delta_{r}(r-1) \\
\text{where }\\
&\delta_{r} =\left(\frac{\delta}{r}\right)\left(\frac{\theta}{\theta-1}\right)
\end{aligned}\]

Now, to simplify the linearisation process, we consider the equation (9)
in parts:

\hypertarget{term-1-1}{%
\subsubsection{Term 1:}\label{term-1-1}}

\[\begin{aligned}
\frac{a_{t}}{\delta}\left[\ln \left(m^{*}\right)-\ln \left(m_{t}\right)+\ln \left(u_{t}\right)\right]
= &\frac{a_{t}}{\delta}\left\{\ln \left(m^{*}\right)-\ln (m)-\hat{m}_{t}+\hat{u}_{t}\right\} \\
= &\frac{a_{t}}{\delta}\left[\delta_{r}(r-1)-\hat{m}_{t}+\hat{u}_{t}\right] \\
= &\frac{a_{t}}{\delta}\delta_{r}(r-1) + \frac{a_{t}}{\delta} \left( \hat{u}_{t}-\hat{m}_{t} \right)
\end{aligned}\]

\hypertarget{term-5}{%
\subsubsection{Term 5:}\label{term-5}}

\[\begin{aligned} \lambda_{t}\left(1-\frac{1}{r_{t}}\right) 
&=\lambda e^{\hat{\lambda}_{t}}\left(1-\frac{e^{-\hat{r}_{t}}}{r}\right)\\
&=\lambda\left[e^{\hat{\lambda}_{t}}-\frac{e^{\lambda_{t}-\hat{r}_{t}}}{r}\right]\\
&=\lambda\left[1+\hat{\lambda}_{t}-\frac{1+\hat{\lambda}_{t}-\hat{r}_{t}}{r}\right]\\
&=\lambda+\hat{\lambda \lambda}_{t} - \lambda/r + \lambda \cdot \hat{\lambda}_{t}/r - \lambda \cdot \hat{r}_{t}/r \end{aligned}\]

Using \(\lambda - \lambda/r = \frac{\delta_r(r-1)}{\delta}\) to add the
previous two solutions together and changing the sign for carrying the
term to the other side of the equality, we get:

\[\begin{aligned}&\lambda+\hat{\lambda \lambda}_{t} - \lambda/r + \lambda \cdot \hat{\lambda}_{t}/r - \lambda \cdot \hat{r}_{t}/r -\frac{a_{t}}{\delta}\delta_{r}(r-1)\\
= &-\frac{e^{\hat{a}_{t}}}{\delta}\delta_{r}(r-1) + \lambda-\lambda/r+\hat{\lambda}_t \cdot \lambda/r \cdot (r-1) + \lambda \cdot \hat{r}_t/r\\
=& -\frac{\delta_r(r-1) \hat{a}_t}{\delta} + \hat{\lambda}\frac{\delta_r (r-1)}{\delta} +\hat{r}_t \frac{\delta_r}{\delta}
\end{aligned}\]

\hypertarget{term-2-1}{%
\subsubsection{Term 2:}\label{term-2-1}}

\[\begin{aligned}&a_{t}\left(\frac{\phi_{m}}{2}\right)\left(\frac{Z_t m_t }{z Z_{t-1}m_{t-1}}-1\right)^{2}\\
=& \left(\frac{\phi_{m}}{2}\right) \left[ \frac{a_{t} z_{t}^{2} m_{t}^{2}}{z^{2} m_{t-1}^{2}} - 2 \frac{a_{t} z_{t} m_{t}}{z m_{t-1}} + a_{t} \right]\\
=& \left(\frac{\phi_{m}}{2}\right) \left[ \exp(\hat{a}_{t}+2 \hat{z}_{t}+2 \hat{m}_{t}-2 \hat{m}_{t-1})-2 \exp(\hat{a}_{t}+\hat{m}_{t}+\hat{z}_{t}-\hat{m}_{t-1})+ \exp(\hat{a}_{t}) \right] \\
=& \left(\frac{\phi_{m}}{2}\right) \left[ 1 + \hat{a}_{t}+2 \hat{z}_{t}+2 \hat{m}_{t}-2 \hat{m}_{t-1} - 2( 1 + \hat{a}_{t}+\hat{m}_{t}+\hat{z}_{t}-\hat{m}_{t-1}) + (1 + \hat{a}_{t}) \right] \\
=& \left(\frac{\phi_{m}}{2}\right) \left[ -1 - \hat{a}_t + 1 + \hat{a}_t \right]\\
=& 0
\end{aligned}\]

\hypertarget{term-3-1}{%
\subsubsection{Term 3:}\label{term-3-1}}

\[\begin{aligned} &-a_{t} \phi_{m}\left(\frac{Z_t m_t }{z Z_{t-1}m_{t-1}} - 1\right)\left(\frac{Z_t m_t }{z Z_{t-1}m_{t-1}}\right)\\
=& e^{\hat{a}_{t}} \phi_{m}\left[\frac{z m e^{\hat{z}_{t}+\hat{m}_{t}}}{z m e^{\hat{m}_{t-1}}}-1\right]\left[\frac{z m e^{\hat{z}_{t}+\hat{m}_{t}}}{z m e^{\hat{m}_{t-1}}}\right]\\
=& \phi_m \left[ \exp \left(2 \hat{z}_{t}+2 \hat{m}_{t}-2 \hat{m}_{t-1}+\hat{a}_{t}\right)-\exp \left(\hat{z}_{t}+\hat{m}_{t}-\hat{m}_{t-1}+\hat{a}_{t}\right) \right]\\
=& \phi_m \left[\hat{z}_{t}+\hat{m}_{t}-\hat{m}_{t-1}\right]
\end{aligned}\]

\hypertarget{term-4-1}{%
\subsubsection{Term 4:}\label{term-4-1}}

\[\begin{aligned}&\beta \phi_{m} E_{t}\left[e^{\hat{a}_{t+1}}\left(\frac{z m e^{\hat{z}_{t+1}+\hat{m}_{t+1}}}{z m e^{\hat{m}_{t}}}-1\right)\left(\frac{z m e^{\hat{z}_{t+1}+\hat{m}_{t+1}}}{z m e^{\hat{m}_{t}}}\right)^{2} \left(\frac{z}{z e^{\hat{z}_{t+1}}}\right)\right] \\
=&\beta \phi_{m} E_{t}\left\{\exp \left(\hat{z}_{t+1}+\hat{m}_{t+1}+\hat{a}_{t+1}-\hat{m}_{t}+2 \hat{z}_{t+1}+2 \hat{m}_{t+1}-2 \hat{m}_{t}-\hat{z}_{t+1}\right)\right. \\
& \left.-\exp \left(\hat{a}_{t+1}+2 \hat{z}_{t+1}+2 \hat{m}_{t+1}-2 \hat{m}_{t}-\hat{z}_{t+1}\right)\right\} \\
=&\beta \phi_{m} E_{t}\left\{\exp \left(2 \hat{z}_{t+1}+3 \hat{m}_{t+1}-3 \hat{m}_{t}+\hat{a}_{t+1}\right)-\exp \left(\hat{z}_{t+1}+2 \hat{m}_{t+1}-2 \hat{m}_{t}+\hat{a}_{t+1}\right)\right\} \\
=&\beta \phi_{m} E_{t}\left\{\hat{z}_{t+1}+\hat{m}_{t+1}-\hat{m}_{t}\right\}\\
=& \beta \phi_{m} E_{t}\left\{\hat{m}_{t+1}\right\} - \beta \phi_m \hat{m}_{t}
\end{aligned}\]

\hypertarget{adding-the-terms-together-for-the-final-solution}{%
\subsubsection{Adding The Terms Together for the Final
Solution:}\label{adding-the-terms-together-for-the-final-solution}}

\[\begin{aligned} &\frac{a_{t}}{\delta} \left( \hat{u}_{t}-\hat{m}_{t} \right) - \phi_m \left[\hat{z}_{t}+\hat{m}_{t}-\hat{m}_{t-1}\right] + \beta \phi_{m} E_{t}\left\{\hat{m}_{t+1}\right\} - \beta \phi_m \hat{m}_{t} \\
= &  -\frac{\delta_r(r-1) \hat{a}_t}{\delta} + \hat{\lambda}\frac{\delta_r (r-1)}{\delta} +\hat{r}_t \frac{\delta_r}{\delta}
\end{aligned}\]

Multiply by \(\delta\) throughout:

\[\begin{aligned} & a_t \left( \hat{u}_{t}-\hat{m}_{t} \right) - \phi \left[\hat{z}_{t} + \hat{m}_{t}-\hat{m}_{t-1}\right] + \beta \phi E_{t}\left\{\hat{m}_{t+1}\right\} - \beta \phi \hat{m}_{t} \\
=&  -\delta_r(r-1) \hat{a}_t + \hat{\lambda}\delta_r (r-1) +\hat{r}_t \delta_r \\\end{aligned}\]

Assuming \(a_t = 1\) the final solution follows:

\[\begin{aligned}\therefore \quad & \delta_r (r-1) \hat{\lambda} - \delta_r(r-1) \hat{a}_t -\hat{u}_{t} +\phi \hat{z}_{t}
= \phi \hat{m}_{t-1}-[1+(1+\beta) \phi] \hat{m}_{t}+\beta \phi \hat{m}_{t+1}-\delta_{r} \hat{r}_{t}
\end{aligned}\]

\hypertarget{equation-26}{%
\subsection{Equation (26)}\label{equation-26}}

\[\mu_{t}=z_{t}\left(m_{t} / m_{t-1}\right) \pi_{t}\]

Applying uhlig's Method:

\[\begin{aligned}
&\mu_{t} =z_{t}\left[\frac{m_{t}}{m_{t-1}}\right] \pi_{t} \\
&\mu e^{\hat{\mu}_{t}}=z e^{\hat{z_{t}}}\left[\frac{m e^{\hat{m}_{t}}}{m e^{\hat{m}_{t-1}}}\right] \pi e^{\hat{\pi}_{t}} \\
\Rightarrow & \left(1+\hat{\mu}_{t}\right) =\left(1+\hat{z}_{t}+\hat{m}_{t}-\hat{m}_{t-1}+\hat{\pi}_{t}\right) \\
\therefore \quad & \hat{\mu}_{t} =\hat{z}_{t}+\hat{m}_{t}-\hat{m}_{t-1}+\pi_{t}
\end{aligned}\]

\hypertarget{equation-27}{%
\subsection{Equation (27)}\label{equation-27}}

This equation

\[g_{t}=\left(y_{t} / y_{t-1}\right) z_{t}\]

Using the Steady State solution

\[g = z\]

\newpage

Yields the following solution, using Uhlig's method:

\[\begin{aligned}
g_{t}&=\left(\frac{y_{t}}{y_{t-1}}\right) z_{t}\\
\Rightarrow ge^{g_{t}}&=\left[\frac{y e^{\hat{y}_{t}}}{y e^{\hat{y}_{t-1}}}\right] z e^{\hat{z}_{t}}\\
\Rightarrow \left(1+\hat{g}_{t}\right)&=\left(1+\hat{y}_{t}-\hat{y}_{t-1}+\hat{z}_{t}\right)\\
\therefore \hat{g}_{t} &= \hat{y}_{t}-\hat{y}_{t-1}+\hat{z}_{t}
\end{aligned}\]

\hypertarget{equation-28}{%
\subsection{Equation (28)}\label{equation-28}}

\[\begin{aligned}
& \ln \left(a_{t}\right)=p_{a} \ln \left(a_{t-1}\right)+\varepsilon_{a t} \\
& \ln \left(a e^{\hat{a}_{t}}\right)=p_{a} \ln \left(a e^{\hat{a}_{t-1}}\right)+\varepsilon_{a t} \\
\Rightarrow & \ln (a)+\hat{a}_{t}=\rho_{a} \ln (a)+p_{a} \hat{a}_{t-1}+\varepsilon_{a t} \\
&[a =1] \\
\therefore \quad & \hat{a}_{t} =\rho_{a} \hat{a}_{t-1}+\varepsilon_{a t}
\end{aligned}\]

\hypertarget{equation-29}{%
\subsection{Equation (29)}\label{equation-29}}

\[\begin{aligned}
\ln(z_t)=\ln(z) + \varepsilon_{z t}\\
\Rightarrow \ln\left(z e^{\hat{z}_t}\right) =\ln (z)+\varepsilon_{z t}\\
\Rightarrow  \ln (z)+\hat{z}_{t}=\ln (z)+\varepsilon_{z t}\\
\therefore \hat{z}_t = \varepsilon_{z t}
\end{aligned}\]

\hypertarget{equation-30}{%
\subsection{Equation (30)}\label{equation-30}}

\[\begin{aligned}
& \ln \left(u_{t}\right)=p_{u} \ln \left(u_{t-1}\right)+\varepsilon_{u t} \\
& \ln \left(u e^{\hat{u}_{t}}\right)=p_{u} \ln \left(u e^{\hat{u}_{t-1}}\right)+\varepsilon_{u t} \\
\Rightarrow & \ln (u)+\hat{u}_{t}=\rho_{u} \ln (u)+p_{u} \hat{u}_{t-1}+\varepsilon_{u t} \\
&[u =1] \\
\therefore \quad &\hat{u}_{t} =\rho_{u} \hat{u}_{t-1}+\varepsilon_{u t}
\end{aligned}\]

\hypertarget{equation-31}{%
\subsection{Equation (31)}\label{equation-31}}

\[\begin{aligned}
& \ln \left(\theta_{t}\right)=\left(1-\rho_{\theta}\right) \ln (\theta)+\rho_{\theta} \ln \left(\theta_{t-1}\right)+\varepsilon_{\theta t} \\
\Rightarrow & \ln \left(\theta e^{\hat{\theta}_{t}}\right)=\left(1-\rho_{\theta}\right) \ln (\theta)+\rho_{\theta} \ln \left(\theta e^{\hat{\theta}{t-1}}\right)+\varepsilon_{\theta t} \\
\Rightarrow & \ln (\theta)+\hat{\theta}_{t}=\ln (\theta)-\rho_{\theta} \ln (\theta)+\rho_{\theta} \ln (\theta)+\rho_{\theta} \hat{\theta}_{t-1}+\varepsilon_{\theta t} \\
\therefore & \quad \hat{\theta}_{t}=\rho_{\theta} \hat{\theta}_{t-1}+\varepsilon_{\theta t}
\end{aligned}\]

using the normalisation
\[\hat{e}_{t}=-\left(1 / \phi_{p}\right) \hat{\theta}_{t}\]

and

\[\rho_{e}=\rho_{\theta}\]

We can rewrite:

\[\begin{aligned}
&\hat{\theta}_{t}=\rho_{\theta} \hat{\theta}_{t-1}+\varepsilon_{\theta t}\\
\Rightarrow - & \left(1 / \phi_{p}\right) \cdot \hat{\theta}_{t} =  \hat{e}_t = -\left(1 / \phi_{p}\right) \left[ \rho_{\theta} \hat{\theta}_{t-1}+\varepsilon_{\theta t} \right]\\
\Rightarrow \quad &\hat{e}_t =  \rho_{e} \hat{e}_{t-1}+\varepsilon_{e t}
\end{aligned}\]

\newpage

\hypertarget{data-and-calibration-methods}{%
\section{Data and Calibration
Methods}\label{data-and-calibration-methods}}

Impulse response functions show the movement of macroeconomic variables
back to steady state in response to a shock. This paper makes use of
these graphs to compare different approaches to monetary policy, namely
the Taylor (TR) rule, the zero Interest Rate (IR) rule, the Flexible
Money Growth (FMG) rule, and the Constant Money Growth (CMG) rule.

To construct these functions, parameter values are inserted into the
log-linearised equations from the previous section. Some of the
parameter values are found via our own calculations. Upon researching
calibrated values of the Phillips Curve slope, we found Hazell et
al.~(2020)'s estimation of the Phillips Curve slope. For the rest,
posterior parameter values from Belongia \& Ireland (2020) are used. We
use the mean posterior values as opposed to the prior values so that all
parameters are estimated values, since calibration is based on estimates
believed to be true. The prior values are those that match the steady
state values. The table below lists the different parameter values used.

TABLE

The steady state growth rate was calculated in accordance with Belongia
\& Ireland (2020)'s method. 1983-2019 quarterly GDP data from FRED is
used, which is then converted into per capita terms by dividing by the
civilian noninstitutional population. The trend of the quarterly GDP per
capita data is then extracted using a Hodrick-Prescott filter with a
smoothing parameter of 1600. After this, the average of the natural log
change in quarterly GDP is calculated as 1.00037. In order to calculate
the steady state interest rate, we use FRED's quarterly interest rate
data between 1983 and 2019. The average of this data is then divided by
four to get 0.98783908.

rho\_pi, rho\_x and rho\_r are only used for the Taylor rule
calibration. Under the FMG and CMG rules, rho\_mm, rho\_mpi and rho\_mx
replace rho\_pi, rho\_x and rho\_r, due to the switch from equation 24
to 33. When calibrating the CMG rule, these parameters are set to zero.
Under the FMG rule, we set rho\_mm, rho\_mpi and rho\_mx equal to 1, 0
and -0.125 respectively. This is in accordance to Belongia \& Ireland
(2020).

To conduct robustness checks for z\_ss, r\_ss, beta, and psi, we replace
them with 1.0046, 1.0121, 0.9987, and 0.1 respectively. The first three
values are taken from Ireland (2010), and we have replaced the value for
the Phillips Curve slope with the mean posterior value from Belongia \&
Ireland (2020). When comparing our impulse response functions to those
using these values, we conclude that they are similar, implying that the
values used are appropriate. The following section compares the impulse
response functions under each rule after a small and positive
productivity, preference, cost push, and money demand shocks are
induced.

\newpage

\hypertarget{results}{%
\section{Results}\label{results}}

\hypertarget{productivity-shock}{%
\subsection{Productivity Shock}\label{productivity-shock}}

The figure below shows the impulse response functions of the output
growth, inflation, nominal interest rate, money growth rate and output
gap after a positive productivity shock. The graphs depict the way in
which each of these macroeconomic variables change under the Taylor (TR)
rule, the Interest Rate (IR) rule, the Flexible Money Growth (FMG) rule
and the Constant Money Growth (CMG) rule.

GRAPH

The impulse response functions (IRF) of the output growth of the TR
rule, IR rule, FMG rule, and CMG rule in response to a productivity
shock are illustrated in the first of the graphs above. The effect of
the TR rule and the IR rule on output growth is identical. The IRF of
output growth under the FMG rule, TR rule and IR rule jumps above 0.002.
Under these three rules, it then slowly moves down to zero at a
decreasing rate. The IRF of output growth under the IR rule moves
faster, and reaches zero after 12 periods, whereas under the FMG rule,
it is more gradual and reaches zero at around 14 periods. Under both
rules, the trajectory of output growth crosses the zero line, but not by
much and moves upwards toward zero again. Under the CMG rule, however,
the IRF starts below zero, and moves up to cross the zero line just past
two periods. It then proceeds to increase but at a decreasing rate until
approximately six periods. Thereafter, it gradually moves back to steady
state. The trajectory of output growth of all three rules represented
here takes more than 20 periods to settle around the steady state.

The IRF of inflation rate under both the CMG rule and FMG rule responds
to a shock in productivity, however it follows different trajectories
toward zero under the two rules. Under the FMG rule, the IRF jumps to
just below steady state, and crosses the zero line at around three
periods. It rises minimally until a turning point at 10 periods,
afterwhich is returns to zero by 20 periods. The trajectory of the
inflation rate under this rule does not stray far from steady state.
Alternatively, the IRF of inflation under the CMG rule experiences a
bigger fluctuation in response to the productivity shock. It jumps down
to approximately -0.0008 and declines further until two periods have
passed. From there, it increases steadily at an increasing rate towards
zero. It does not, however, reach zero within 20 periods, as the IRF
under the FMG rule does.

Similar to the inflation rate, nominal interest rates under both the CMG
rule and FMG rule respond to a shock in productivity. Both IRFs start
above 0.0008. The IRF under the CMG rule declines steadily at an
increasing rate towards zero, but under the FMG rule it first increases
slightly before starting to decline after approximately two periods. The
trajectory of the nominal interest rate then crosses the zero line just
before 14 periods. At around 18 periods the IRF starts moving back to
zero. Although the graph shows that the trajectory of the nominal
interest rate moves close to zero under both scenarios, it takes more
than 20 periods to reach steady state.

The nominal money growth rate under the TR rule and IR rule in response
to a productivity shock have identical IRFs. Both jump to above 0.001
and decline steadily to reach zero by 20 periods. Alternatively, the IRF
under the FMG rule increases at an increasing rate, after reaching a
point of inflection after about five periods, after which it increases
at a decreasing rate. The trajectory of the money growth rate reaches a
turning point between eight and nine periods, after which it declines to
just above steady state by 20 periods.

In response to a productivity shock, the IRF of output gap under the FMG
rule first declines, and then returns to steady state by eight periods.
It then increases slightly, but almost reaches steady state again after
20 periods have passed. The output gap under the CMG rule behaves
differently in response to such a shock. The IRF also declines at first,
but from a starting point of -0.004 (as opposed to -0.001). Just after
four periods, it reaches a turning point from which it increases at a
decreasing rate. The trajectory of the output gap does not reach steady
state within 20 periods.

Contractionary monetary policy refers to increases in interest rates so
as to decrease the supply of money in the economy in order to lower
inflation (Chen, 2020). After a small positive shock to productivity,
the CMG rule displays signs of this policy, with nominal interest rates
jumping up before slowly returning to steady state. The FMG rule results
in an increase in output growth, after which a rise in the money growth
and a decrease in the nominal interest rate allows for the output growth
to return to steady state without significant changes to inflation.
Under both the TR rule and the IR rule, money growth increases to
accommodate the increase in output, so that the output gap remains
stable. It appears as though the CMG responds the least efficiently to
the shock, with large fluctuations in inflation, as well as decreases in
output growth, and a negative output gap.

\hypertarget{preference-shock}{%
\subsection{Preference Shock}\label{preference-shock}}

The figure below shows the response of the chosen macroeconomic
variables to a positive preference shock. As explained above, each panel
shows the response of the given variable under each of the four rules. A
preference shock refers to a shock to the utility aspect of the economy
(Chugh, 2015:147). As such, it affects the macroeconomic variables from
the demand side (Malik et al., 2019:12).

The output growth of all four scenarios starts above zero in response to
a preference shock, in order of IR rule, TR Rule, FMG rule and CMG rule
from lowest to highest. The trajectory of the output growth under all
scenarios then declines, however, under the TR Rule, FMG rule and CMG
rule, it becomes negative. Under the TR rule, the IRF stays very close
to zero. The trajectory of the output growth of the FMG and CMG rules
decrease further below zero but start to move back to steady state
between six and eight periods. The IRF under the FMG rule reaches zero
around 12 periods, whereas under the CMG rule, it increases more
gradually, nearing it at 20 periods. The IRF of output growth under the
IR rule experienced the smallest jump up after the shock and returned to
steady state after four periods.

Inflation under the CMG rule jumps to above 0.0002 in response to a
preference shock. The IRF proceeds to increase further for two periods,
after which it declines steadily past zero. It is still declining after
18 periods. The IRFs under both the FMG rule and the TR rule experience
a small jump and start between zero and 0.0001. Under the TR rule, the
IRF remains close to steady state, and gradually proceeds to zero by 20
periods. Under the FMG rule, it sharply declines to cross the zero line
and reaches a turning point between seven and eight periods. It
increases to cross the zero line again at around 17 periods. It remains
slightly above steady state after 20 periods.

The nominal interest rate under the TR rule does not experience an
immediate jump in response to a preference shock. Instead, the IRF
slowly increases up to 0.0003 until it starts to decrease after six
periods. It slowly moves down towards zero, but does not reach steady
state within 20 periods. The nominal interest rate under the FMG rule
and CMG rule both jump down minimally in response to a preference shock.
Starting at -0.0001, however, the IRF of the nominal interest rate under
the FMG rule increases and then plateaus below 0.0001, whereas that of
the CMG rule remains level for approximately two periods, after which it
starts to increase. After 14 periods, it reaches a turning point where
it starts to return to zero, however, it does not reach steady state
within 20 periods.

The money growth rate under the FMG rule does not jump immediately after
a preference shock. The IRF declines from zero at a decreasing rate. It
reaches a turning point between seven and eight periods after the shock,
after which it increases steadily. The IRF of the money growth rate
crosses the zero line just before 16 periods and proceeds to rise. It is
still moving away from steady state at 20 periods. The money growth rate
in response to the shock under the TR rule jumps to below -0.004 and
remains there for approximately three periods. It then starts increasing
and returns to steady state between nine and 10 periods, after which it
moves away from steady state again in the positive direction. Unlike the
money growth rate under the FMG rule, however, it starts moving back
towards the steady state before 20 periods. It does not reach steady
state within 20 periods.

The output gap under the TR rule does not deviate from steady state
significantly in response to a preference shock. The IRF experiences a
small positive jump, but returns to steady state between 10 and 12
periods. The trajectory of the output gap under the FMG rule jumps up
slightly more than under the TR rule, but instead of returning to zero
immediately, it increases first. After two periods, it starts to
decrease, and crosses the zero line after approximately seven periods.
After this point, it deviates slightly negatively from steady state, but
slowly moves back. By 20 periods, it has returned close to steady state.

The FMG rule displays a delayed response to the preference shock, with
eventual contractionary policy. The nominal interest rate first
decreases, after which it increases significantly. Along with this
increase in the interest rate, the money growth rate declines. As a
result, inflation minimally decreases, and the output growth and gap
remain stable. The same cannot be said, however, for the CMG rule, where
the money growth does not change to accomodate for the shock. Under this
rule, inflation responds in a volatile manner. Additionally, the output
growth and gap experience the biggest jumps under this rule. Under the
TR rule, money growth decreases before it starts to increase. Interest
rates do not respond immediately, but eventually increase. These changes
result in inflation and the output gap being kept stable. The jump in
output growth is also minimal. Macroeconomic factors under the IR rule
do respond strongly to a preference shock. There is only a very small
positive change in output growth.

\hypertarget{money-demand-shock}{%
\subsection{Money Demand Shock}\label{money-demand-shock}}

The figure below shows the response of the chosen macroeconomic
variables to a positive money demand shock. As explained above, each
panel shows the response of the given variable under each of the four
rules. According to Lorenzoni (2006), output volatility may be
attributed to demand shocks. A positive money demand shock increases
households' spending, leading to a rise in inflation.

As seen in the first impulse response function above, there is a
negative effect on output growth under both the CMG rate and FMG rate
rules after a positive money demand shock. The IRF of output growth
starts at approximately -0.001 under the FMG rate rule and approximately
-0.003 under the CMG rate rule. The IRF increases at a decreasing rate
under both rules, with the trajectory of output growth under the FMG
rate rule crossing the zero line first. Under both, the IRFs begin
decreasing, with the FMG rate rule reaching the steady state at 14
periods, but then proceeds to slightly decrease below the steady state
thereafter. Under the CMG rate rule, the IRF does not reach the steady
state within twenty periods.

Inflation under both the CMG rate rule and the FMG rate rule is impacted
negatively by the money demand shock. The negative effect under the CMG
rate rule is larger than under the FMG rate rule. Under the CMG rate
rule, the IRF starts at -0.0005 and approximately -0.00005 under the FMG
rate rule. Under the CMG rate rule, the IRF initially decreases before
increasing at two periods. It increases at a decreasing rate and reaches
the steady state at 18 periods, before surpassing the steady state and
continues to increase. Under the FMG rate rule, the IRF increases until
eight periods, then slightly decreases. The trajectory of Inflation
reaches, and remains at, the steady state after 18 periods.

The nominal interest rate under both the FMG rate rule and CMG rate rule
experiences a positive shock and the IRFs start at 0.0008. Initially,
the IRF under the CMG rate rule declines at a faster rate than under the
FMG rate rule. However, at seven periods, the IRF under the FMG rate
rule drops below the CMG rate rule. Under the FMG rate rule, the IRF of
the nominal interest rate reaches the steady state at 13 periods but
continues to decrease at a decreasing rate. Under the CMG rate rule, the
IRF does not reach the steady state within 20 periods.

As seen in the graph above, after the shock, the money growth rate is
identically impacted by the TR rule and the IR rule. There is an initial
positive effect on the money growth rate, however the IRF begins to
decline, where it meets the steady state at 10 periods. Thereafter, the
IRF of the money growth rate drops below the steady state and continues
to decrease at a decreasing rate. Under the FMG rule, the IRF of the
money growth rate starts at zero and initially increases. After nine
periods, the IRF begins to decrease and meets the steady state after 18
periods, thereafter, dropping below the steady state.

The output gap is negatively affected by the shock under both the
constant money growth rate rule and the flexible money growth rate rule.
Under the FMG rate rule, the IRF of the output gap starts at -0.001 and
initially decreases. After three periods, the IRF increases to meet the
steady state at eight periods. It increases at a decreasing rate and
begins to stabilise after 13 periods. Under the CMG rate rule, the IRF
of the output gap starts at 0.003 and decreases until the 4th period. It
then increases at a decreasing rate and does not reach the steady state
within 20 periods.

\hypertarget{cost-push-shock}{%
\subsection{Cost Push Shock}\label{cost-push-shock}}

The figure below shows the response of the chosen macroeconomic
variables to a positive cost push shock. As explained above, each panel
shows the response of the given variable under each of the four rules. A
cost push shock results in higher inflation due to an increase in the
price of production, causing a decrease in aggregate supply.

Under the four rules, output growth is not significantly affected by the
cost push shock. However, under the IR rule, output growth is positively
affected, as the IRF starts at 0,01 and decreases to meet the steady
state after approximately 12 periods.

Under the TR rule, the FMG rate rule and the CMG rate rule, there is a
positive effect of the cost push shock on inflation, which is to be
expected. Under the FMG rule, the IRF of inflation starts at 0.015 and
sharply decreases for the first three periods, thereafter, decreases
more subtly until reaching the steady state after 12 periods. Under the
CMG rule, the IRF starts at 0.00125 and decreases to surpass the zero
line before four periods. It then increases but does not reach the
steady state within 20 periods. Under the TR, the IRF starts at 0.0025
and decreases to meet the steady state after 4 periods.

The nominal interest rate is affected by the shock under the CMG rate
rule and the FMG rate rule in a similar fashion, as a response to the
shock. Under both, there is a positive effect of the shock on nominal
interest rate, where the IRF starts at 0.008. It then declines at a
faster rate under the FMG rate rule initially, but after six periods,
the IRF declines at a faster rate under the CMG rate rule. Under the CMG
rate rule, the IRF reaches the steady state sooner, at 13 periods, than
the FMG, which does not reach the steady state within 20 periods. Under
the TR, the IRF starts at zero and slightly increases before decreasing
and merging with the steady state after eight periods.

There is a positive effect on the money growth rate under the IR rule
and the TR rule. Under the IR rule, the IRF of the money demand growth
rate starts at approximately 0.003 and decreases sharply to below the
steady state after two periods. At the 4th period, the IRF increases to
meet the steady state after 18 periods. Under the TR rule, the positive
effect is smaller, as the IRF of the money growth rate starts at 0.0015.
It then decreases to meet the steady state after three periods but
increases again. After eight periods, the IRF decreases once again to
meet the steady state after 18 periods. Under the FMG rate rule, the IRF
starts at zero and steadily increases before decreasing after nine
periods. The IRF of the money growth rate does not reach the steady
state within 20 periods.

Under the TR rule, there is no significant impact of the shock on the
output gap. Under the IR rule, the CMG rate rule and the FMG rate rule,
there is a negative impact of the cost push shock on the output gap. The
smallest effect on the output gap is under the IR rule where the IRF of
the output gap starts at just below zero and initially increases to
cross the zero line. It then starts to decrease after 14 periods and
moves towards the steady state, however, it does not reach it within 20
periods. Under the CMG rate rule, the IRF of the output gap starts at
approximately -0.01 and initially decreases before increasing after four
periods. It does not reach the steady state within 20 periods. Under the
IR rule, the IRF of the output gap starts at -0.035 and remains there
until two periods have passed. After that point, it increases at a
decreasing rate and meets the steady state after 14 periods.

\newpage

\hypertarget{conclusion}{%
\section{Conclusion}\label{conclusion}}

\bibliography{Tex/ref}





\end{document}
